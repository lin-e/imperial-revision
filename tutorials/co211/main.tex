\documentclass[a4paper, 12pt]{article}

% packages
\usepackage{amssymb}
\usepackage[fleqn]{mathtools}
\usepackage{tikz}
\usepackage{enumerate}
\usepackage{bussproofs}
\usepackage{xcolor}
\usepackage[margin=1.3cm]{geometry}
\usepackage{logicproof}
\usepackage{diagbox}
\usepackage{listings}
\usepackage{graphicx}
\usepackage{lstautogobble}
\usepackage{hyperref}
\usepackage{multirow}
\usepackage{tipa}
\usepackage{pgfplots}
\usepackage{adjustbox}

% tikz libraries
\usetikzlibrary{
    decorations.pathreplacing,
    arrows,
    shapes.gates.logic.US,
    circuits.logic.US,
    calc,
    automata,
    positioning,
    intersections
}

\pgfplotsset{compat=1.16}

\pgfmathdeclarefunction{gauss}{2}{%
  \pgfmathparse{1/(#2*sqrt(2*pi))*exp(-((x-#1)^2)/(2*#2^2))}%
}

\allowdisplaybreaks % allow environments to break
\setlength\parindent{0pt} % no indent

% shorthand for verbatim
% this clashes with logicproof, so maybe fix this at some point?
\catcode`~=\active
\def~#1~{\texttt{#1}}

% code listing
\lstdefinestyle{main}{
    numberstyle=\tiny,
    breaklines=true,
    showspaces=false,
    showstringspaces=false,
    tabsize=2,
    numbers=left,
    basicstyle=\ttfamily,
    columns=fixed,
    fontadjust=true,
    basewidth=0.5em,
    autogobble,
    xleftmargin=3.0ex,
    mathescape=true
}
\newcommand{\dollar}{\mbox{\textdollar}} %
\lstset{style=main}

% augmented matrix
\makeatletter
\renewcommand*\env@matrix[1][*\c@MaxMatrixCols c]{%
\hskip -\arraycolsep
\let\@ifnextchar\new@ifnextchar
\array{#1}}
\makeatother

% ceiling / floor
\DeclarePairedDelimiter{\ceil}{\lceil}{\rceil}
\DeclarePairedDelimiter{\floor}{\lfloor}{\rfloor}

% custom commands
\newcommand{\indefint}[2]{\int #1 \, \mathrm{d}#2}
\newcommand{\defint}[4]{\int_{#1}^{#2} #3 \, \mathrm{d}#4}
\newcommand{\pdif}[2]{\frac{\partial #1}{\partial #2}}
\newcommand{\dif}[2]{\frac{\mathrm{d}#1}{\mathrm{d}#2}}
\newcommand{\limit}[2]{\raisebox{0.5ex}{\scalebox{0.8}{$\displaystyle{\lim_{#1 \to #2}}$}}}
\newcommand{\limitsup}[2]{\raisebox{0.5ex}{\scalebox{0.8}{$\displaystyle{\limsup_{#1 \to #2}}$}}}
\newcommand{\summation}[2]{\sum\limits_{#1}^{#2}}
\newcommand{\product}[2]{\prod\limits_{#1}^{#2}}
\newcommand{\intbracket}[3]{\left[#3\right]_{#1}^{#2}}
\newcommand{\laplace}{\mathcal{L}}
\newcommand{\fourier}{\mathcal{F}}
\newcommand{\mat}[1]{\boldsymbol{#1}}
\renewcommand{\vec}[1]{\boldsymbol{#1}}
\newcommand{\rowt}[1]{\begin{bmatrix}
    #1
\end{bmatrix}^\top}
\DeclareMathOperator*{\argmax}{argmax}
\DeclareMathOperator*{\argmin}{argmin}

\newcommand{\lto}[0]{\leadsto\ }

\newcommand{\ulsmash}[1]{\underline{\smash{#1}}}

\newcommand{\powerset}[0]{\wp}
\renewcommand{\emptyset}[0]{\varnothing}

\makeatletter
\newsavebox{\@brx}
\newcommand{\llangle}[1][]{\savebox{\@brx}{\(\m@th{#1\langle}\)}%
  \mathopen{\copy\@brx\kern-0.5\wd\@brx\usebox{\@brx}}}
\newcommand{\rrangle}[1][]{\savebox{\@brx}{\(\m@th{#1\rangle}\)}%
  \mathclose{\copy\@brx\kern-0.5\wd\@brx\usebox{\@brx}}}
\makeatother
\newcommand{\lla}{\llangle}
\newcommand{\rra}{\rrangle}
\newcommand{\la}{\langle}
\newcommand{\ra}{\rangle}
\newcommand{\crnr}[1]{\text{\textopencorner} #1 \text{\textcorner}}
\newcommand{\bnfsep}[0]{\ |\ }
\newcommand{\concsep}[0]{\ ||\ }

\newcommand{\axiom}[1]{\AxiomC{#1}}
\newcommand{\unary}[1]{\UnaryInfC{#1}}
\newcommand{\binary}[1]{\BinaryInfC{#1}}
\newcommand{\trinary}[1]{\TrinaryInfC{#1}}
\newcommand{\quaternary}[1]{\QuaternaryInfC{#1}}
\newcommand{\quinary}[1]{\QuinaryInfC{#1}}
\newcommand{\dproof}[0]{\DisplayProof}
\newcommand{\llabel}[1]{\LeftLabel{\scriptsize #1}}
\newcommand{\rlabel}[1]{\RightLabel{\scriptsize #1}}

\newcommand{\ttbs}{\char`\\}
\newcommand{\lrbt}[0]{\ \bullet\ }

% colours
\newcommand{\violet}[1]{\textcolor{violet}{#1}}
\newcommand{\blue}[1]{\textcolor{blue}{#1}}
\newcommand{\red}[1]{\textcolor{red}{#1}}
\newcommand{\teal}[1]{\textcolor{teal}{#1}}

% reasoning proofs
\usepackage{ltablex}
\usepackage{environ}
\keepXColumns
\NewEnviron{reasoning}{
    \begin{tabularx}{\textwidth}{rlX}
        \BODY
    \end{tabularx}
}
\newcommand{\proofline}[3]{$(#1)$ & $#2$ & \hfill #3 \smallskip \\}
\newcommand{\proofarbitrary}[1]{& take arbitrary $#1$ \smallskip \\}
\newcommand{\prooftext}[1]{\multicolumn{3}{l}{#1} \smallskip \\}
\newcommand{\proofmath}[3]{$#1$ & = $#2$ & \hfill #3 \smallskip \\}
\newcommand{\prooftherefore}[1]{& $\therefore #1$ \smallskip \\}
\newcommand{\proofbc}[0]{\prooftext{\textbf{Base Case}}}
\newcommand{\proofis}[0]{\prooftext{\textbf{Inductive Step}}}

% ER diagrams
\newcommand{\nattribute}[4]{
    \node[draw, state, inner sep=0cm, minimum size=0.2cm, label=#3:{#4}] (#1) at (#2) {};
}
\newcommand{\mattribute}[4]{
    \node[draw, state, accepting, inner sep=0cm, minimum size=0.2cm, label=#3:{#4}] (#1) at (#2) {};
}
\newcommand{\dattribute}[4]{
    \node[draw, state, dashed, inner sep=0cm, minimum size=0.2cm, label=#3:{#4}] (#1) at (#2) {};
}
\newcommand{\entity}[3]{
    \node[] (#1-c) at (#2) {#3};
    \node[inner sep=0cm] (#1-l) at ($(#1-c) + (-1, 0)$) {};
    \node[inner sep=0cm] (#1-r) at ($(#1-c) + (1, 0)$) {};
    \node[inner sep=0cm] (#1-u) at ($(#1-c) + (0, 0.5)$) {};
    \node[inner sep=0cm] (#1-d) at ($(#1-c) + (0, -0.5)$) {};
    \draw
    ($(#1-c) + (-1, 0.5)$) -- ($(#1-c) + (1, 0.5)$) -- ($(#1-c) + (1, -0.5)$) -- ($(#1-c) + (-1, -0.5)$) -- cycle;
}
\newcommand{\relationship}[3]{
    \node[] (#1-c) at (#2) {#3};
    \node[inner sep=0cm] (#1-l) at ($(#1-c) + (-1, 0)$) {};
    \node[inner sep=0cm] (#1-r) at ($(#1-c) + (1, 0)$) {};
    \node[inner sep=0cm] (#1-u) at ($(#1-c) + (0, 1)$) {};
    \node[inner sep=0cm] (#1-d) at ($(#1-c) + (0, -1)$) {};
    \draw
    ($(#1-c) + (-1, 0)$) -- ($(#1-c) + (0, 1)$) -- ($(#1-c) + (1, 0)$) -- ($(#1-c) + (0, -1)$) -- cycle;
}

% AVL Trees
\newcommand{\avltri}[4]{
    \draw ($(#1)$) -- ($(#1) + #4*(0.5, -1)$) -- ($(#1) + #4*(-0.5, -1)$) -- cycle;
    \node at ($(#1) + #4*(0, -1) + (0, 0.5)$) {#3};
    \node at ($(#1) + #4*(0, -1) + (0, -0.5)$) {#2};
}

% RB Trees
\tikzset{rbtr/.style={inner sep=2pt, circle, draw=black, fill=red}}
\tikzset{rbtb/.style={inner sep=2pt, circle, draw=black, fill=black}}

% actual document
\begin{document}
    \section*{CO211 - Operating Systems \hfill Tutorial Sheets}
        \subsection*{Tutorial 1 - Introduction}
            \begin{enumerate}[1.]
                \itemsep0em
                \item
                    The issue of resource allocation shows up in different forms in different types of operating systems.
                    List the most important resources that must be managed by an operating system in the following settings;
                    \begin{enumerate}[(a)]
                        \itemsep0em
                        \item Supercomputer
                            \medskip

                            Since this is most likely used for computation, processor time as well as memory should be carefully managed.
                        \item Workstations connected to servers via a network
                            \medskip

                            Network access and bandwidth.
                        \item Smartphone
                            \medskip

                            Energy, since it is a portable device, as well as access to hardware such as the camera, GPS, as well as connectivity (Bluetooth, mobile network, etc).
                    \end{enumerate}
                \item What is the kernel of an operating system?
                    \medskip

                    The kernel of the operating system is part of the operating system that remains in memory, executing in the privileged part of the CPU.
                \item
                    Why is the separation into a user mode and a kernel mode considered good operating system design?
                    Give an example in which the execution of a user process switches from user mode to kernel mode, and then back to user mode again.
                    \medskip

                    Any bugs executing in user space should not cause the entire system to crash, since the kernel allows for recovery.
                    If all programs were to run in kernel mode, a failure would bring down the entire system.
                    An example of this switch would be writing to disk (or any system call in general).
                \item Which of the following instructions should only be allowed in kernel mode, and why?
                    \begin{enumerate}[(a)]
                        \itemsep0em
                        \item Disable all interrupts \hfill kernel only
                            \medskip

                            If something in user were to disable interrupts, the kernel would have no way of regaining control,
                        \item Read the time of day clock \hfill user
                            \medskip

                            All user processes should be able to access the time if needed.
                        \item Change the memory map \hfill kernel only
                            \medskip

                            Managing memory should be restricted to the kernel.
                        \item Set the time of day \hfill kernel only
                            \medskip

                            Processes running in user space should not be able to change the time, as it can cause issues for other processes.
                    \end{enumerate}
                \item
                    A portable operating system is one that can be ported from one system architecture to another with little modification.
                    Explain why it is infeasible to build an operating system that is portable without any modification.
                    Describe two general parts that you can find in an operating system that has been designed to be highly portable.
                    \medskip

                    Since the operating system must interact with the hardware, it's not feasible to build an OS that can interact with every hardware configuration.
                    Device drivers (\textbf{platform specific}) allow for the operating system to interact with hardware, and this can be provided by the hardware manufacturer.
                    Another part would be an API the OS provides to programs (\textbf{platform independent}), allowing them to interact with hardware via this abstraction.
            \end{enumerate}
        \subsection*{Tutorial 2 - Processes + Threads}
            \begin{enumerate}[1.]
                \itemsep0em
                \item 
            \end{enumerate}
        \subsection*{Tutorial 3 - Scheduling}
            \begin{enumerate}[1.]
                \itemsep0em
                \item 
            \end{enumerate}
        \subsection*{Tutorial 4 - Synchronisation}
            \begin{enumerate}[1.]
                \itemsep0em
                \item 
            \end{enumerate}
        \subsection*{Tutorial 5 - Deadlocks}
            \begin{enumerate}[1.]
                \itemsep0em
                \item 
            \end{enumerate}
        \subsection*{Tutorial 6 - Memory Management}
            \begin{enumerate}[1.]
                \itemsep0em
                \item 
            \end{enumerate}
        \subsection*{Tutorial 7 - Device Management}
            \begin{enumerate}[1.]
                \itemsep0em
                \item 
            \end{enumerate}
        \subsection*{Tutorial 8 - Disk Management}
            \begin{enumerate}[1.]
                \itemsep0em
                \item 
            \end{enumerate}
        \subsection*{Tutorial 9 - File Systems}
            \begin{enumerate}[1.]
                \itemsep0em
                \item 
            \end{enumerate}
        \subsection*{Tutorial 10 - Security}
            \begin{enumerate}[1.]
                \itemsep0em
                \item 
            \end{enumerate}
\end{document}