\documentclass[a4paper, 12pt]{article}

% packages
\usepackage{amssymb}
\usepackage[fleqn]{mathtools}
\usepackage{tikz}
\usepackage{enumerate}
\usepackage{bussproofs}
\usepackage{xcolor}
\usepackage[margin=1.3cm]{geometry}
\usepackage{logicproof}
\usepackage{diagbox}
\usepackage{listings}
\usepackage{graphicx}
\usepackage{lstautogobble}
\usepackage{hyperref}
\usepackage{multirow}
\usepackage{tipa}
\usepackage{pgfplots}
\usepackage{adjustbox}

% tikz libraries
\usetikzlibrary{
    decorations.pathreplacing,
    arrows,
    shapes.gates.logic.US,
    circuits.logic.US,
    calc,
    automata,
    positioning,
    intersections
}

\pgfplotsset{compat=1.16}

\pgfmathdeclarefunction{gauss}{2}{%
  \pgfmathparse{1/(#2*sqrt(2*pi))*exp(-((x-#1)^2)/(2*#2^2))}%
}

\allowdisplaybreaks % allow environments to break
\setlength\parindent{0pt} % no indent

% shorthand for verbatim
% this clashes with logicproof, so maybe fix this at some point?
\catcode`~=\active
\def~#1~{\texttt{#1}}

% code listing
\lstdefinestyle{main}{
    numberstyle=\tiny,
    breaklines=true,
    showspaces=false,
    showstringspaces=false,
    tabsize=2,
    numbers=left,
    basicstyle=\ttfamily,
    columns=fixed,
    fontadjust=true,
    basewidth=0.5em,
    autogobble,
    xleftmargin=3.0ex,
    mathescape=true
}
\newcommand{\dollar}{\mbox{\textdollar}} %
\lstset{style=main}

% augmented matrix
\makeatletter
\renewcommand*\env@matrix[1][*\c@MaxMatrixCols c]{%
\hskip -\arraycolsep
\let\@ifnextchar\new@ifnextchar
\array{#1}}
\makeatother

% ceiling / floor
\DeclarePairedDelimiter{\ceil}{\lceil}{\rceil}
\DeclarePairedDelimiter{\floor}{\lfloor}{\rfloor}

% custom commands
\newcommand{\indefint}[2]{\int #1 \, \mathrm{d}#2}
\newcommand{\defint}[4]{\int_{#1}^{#2} #3 \, \mathrm{d}#4}
\newcommand{\pdif}[2]{\frac{\partial #1}{\partial #2}}
\newcommand{\dif}[2]{\frac{\mathrm{d}#1}{\mathrm{d}#2}}
\newcommand{\limit}[2]{\raisebox{0.5ex}{\scalebox{0.8}{$\displaystyle{\lim_{#1 \to #2}}$}}}
\newcommand{\limitsup}[2]{\raisebox{0.5ex}{\scalebox{0.8}{$\displaystyle{\limsup_{#1 \to #2}}$}}}
\newcommand{\summation}[2]{\sum\limits_{#1}^{#2}}
\newcommand{\product}[2]{\prod\limits_{#1}^{#2}}
\newcommand{\intbracket}[3]{\left[#3\right]_{#1}^{#2}}
\newcommand{\laplace}{\mathcal{L}}
\newcommand{\fourier}{\mathcal{F}}
\newcommand{\mat}[1]{\boldsymbol{#1}}
\renewcommand{\vec}[1]{\boldsymbol{#1}}
\newcommand{\rowt}[1]{\begin{bmatrix}
    #1
\end{bmatrix}^\top}
\DeclareMathOperator*{\argmax}{argmax}
\DeclareMathOperator*{\argmin}{argmin}

\newcommand{\lto}[0]{\leadsto\ }

\newcommand{\ulsmash}[1]{\underline{\smash{#1}}}

\newcommand{\powerset}[0]{\wp}
\renewcommand{\emptyset}[0]{\varnothing}

\makeatletter
\newsavebox{\@brx}
\newcommand{\llangle}[1][]{\savebox{\@brx}{\(\m@th{#1\langle}\)}%
  \mathopen{\copy\@brx\kern-0.5\wd\@brx\usebox{\@brx}}}
\newcommand{\rrangle}[1][]{\savebox{\@brx}{\(\m@th{#1\rangle}\)}%
  \mathclose{\copy\@brx\kern-0.5\wd\@brx\usebox{\@brx}}}
\makeatother
\newcommand{\lla}{\llangle}
\newcommand{\rra}{\rrangle}
\newcommand{\la}{\langle}
\newcommand{\ra}{\rangle}
\newcommand{\crnr}[1]{\text{\textopencorner} #1 \text{\textcorner}}
\newcommand{\bnfsep}[0]{\ |\ }
\newcommand{\concsep}[0]{\ ||\ }

\newcommand{\axiom}[1]{\AxiomC{#1}}
\newcommand{\unary}[1]{\UnaryInfC{#1}}
\newcommand{\binary}[1]{\BinaryInfC{#1}}
\newcommand{\trinary}[1]{\TrinaryInfC{#1}}
\newcommand{\quaternary}[1]{\QuaternaryInfC{#1}}
\newcommand{\quinary}[1]{\QuinaryInfC{#1}}
\newcommand{\dproof}[0]{\DisplayProof}
\newcommand{\llabel}[1]{\LeftLabel{\scriptsize #1}}
\newcommand{\rlabel}[1]{\RightLabel{\scriptsize #1}}

\newcommand{\ttbs}{\char`\\}
\newcommand{\lrbt}[0]{\ \bullet\ }

% colours
\newcommand{\violet}[1]{\textcolor{violet}{#1}}
\newcommand{\blue}[1]{\textcolor{blue}{#1}}
\newcommand{\red}[1]{\textcolor{red}{#1}}
\newcommand{\teal}[1]{\textcolor{teal}{#1}}

% reasoning proofs
\usepackage{ltablex}
\usepackage{environ}
\keepXColumns
\NewEnviron{reasoning}{
    \begin{tabularx}{\textwidth}{rlX}
        \BODY
    \end{tabularx}
}
\newcommand{\proofline}[3]{$(#1)$ & $#2$ & \hfill #3 \smallskip \\}
\newcommand{\proofarbitrary}[1]{& take arbitrary $#1$ \smallskip \\}
\newcommand{\prooftext}[1]{\multicolumn{3}{l}{#1} \smallskip \\}
\newcommand{\proofmath}[3]{$#1$ & = $#2$ & \hfill #3 \smallskip \\}
\newcommand{\prooftherefore}[1]{& $\therefore #1$ \smallskip \\}
\newcommand{\proofbc}[0]{\prooftext{\textbf{Base Case}}}
\newcommand{\proofis}[0]{\prooftext{\textbf{Inductive Step}}}

% ER diagrams
\newcommand{\nattribute}[4]{
    \node[draw, state, inner sep=0cm, minimum size=0.2cm, label=#3:{#4}] (#1) at (#2) {};
}
\newcommand{\mattribute}[4]{
    \node[draw, state, accepting, inner sep=0cm, minimum size=0.2cm, label=#3:{#4}] (#1) at (#2) {};
}
\newcommand{\dattribute}[4]{
    \node[draw, state, dashed, inner sep=0cm, minimum size=0.2cm, label=#3:{#4}] (#1) at (#2) {};
}
\newcommand{\entity}[3]{
    \node[] (#1-c) at (#2) {#3};
    \node[inner sep=0cm] (#1-l) at ($(#1-c) + (-1, 0)$) {};
    \node[inner sep=0cm] (#1-r) at ($(#1-c) + (1, 0)$) {};
    \node[inner sep=0cm] (#1-u) at ($(#1-c) + (0, 0.5)$) {};
    \node[inner sep=0cm] (#1-d) at ($(#1-c) + (0, -0.5)$) {};
    \draw
    ($(#1-c) + (-1, 0.5)$) -- ($(#1-c) + (1, 0.5)$) -- ($(#1-c) + (1, -0.5)$) -- ($(#1-c) + (-1, -0.5)$) -- cycle;
}
\newcommand{\relationship}[3]{
    \node[] (#1-c) at (#2) {#3};
    \node[inner sep=0cm] (#1-l) at ($(#1-c) + (-1, 0)$) {};
    \node[inner sep=0cm] (#1-r) at ($(#1-c) + (1, 0)$) {};
    \node[inner sep=0cm] (#1-u) at ($(#1-c) + (0, 1)$) {};
    \node[inner sep=0cm] (#1-d) at ($(#1-c) + (0, -1)$) {};
    \draw
    ($(#1-c) + (-1, 0)$) -- ($(#1-c) + (0, 1)$) -- ($(#1-c) + (1, 0)$) -- ($(#1-c) + (0, -1)$) -- cycle;
}

% AVL Trees
\newcommand{\avltri}[4]{
    \draw ($(#1)$) -- ($(#1) + #4*(0.5, -1)$) -- ($(#1) + #4*(-0.5, -1)$) -- cycle;
    \node at ($(#1) + #4*(0, -1) + (0, 0.5)$) {#3};
    \node at ($(#1) + #4*(0, -1) + (0, -0.5)$) {#2};
}

% RB Trees
\tikzset{rbtr/.style={inner sep=2pt, circle, draw=black, fill=red}}
\tikzset{rbtb/.style={inner sep=2pt, circle, draw=black, fill=black}}

% actual document
\begin{document}
    \section*{CO211 - Operating Systems \hfill Tutorial Sheets}
        \subsection*{Tutorial 1 - Introduction}
            \begin{enumerate}[1.]
                \itemsep0em
                \item
                    The issue of resource allocation shows up in different forms in different types of operating systems.
                    List the most important resources that must be managed by an operating system in the following settings;
                    \begin{enumerate}[(a)]
                        \itemsep0em
                        \item Supercomputer
                            \medskip

                            Since this is most likely used for computation, processor time as well as memory should be carefully managed.
                        \item Workstations connected to servers via a network
                            \medskip

                            Network access and bandwidth.
                        \item Smartphone
                            \medskip

                            Energy, since it is a portable device, as well as access to hardware such as the camera, GPS, as well as connectivity (Bluetooth, mobile network, etc).
                    \end{enumerate}
                \item What is the kernel of an operating system?
                    \medskip

                    The kernel of the operating system is part of the operating system that remains in memory, executing in the privileged part of the CPU.
                \item
                    Why is the separation into a user mode and a kernel mode considered good operating system design?
                    Give an example in which the execution of a user process switches from user mode to kernel mode, and then back to user mode again.
                    \medskip

                    Any bugs executing in user space should not cause the entire system to crash, since the kernel allows for recovery.
                    If all programs were to run in kernel mode, a failure would bring down the entire system.
                    An example of this switch would be writing to disk (or any system call in general).
                \item Which of the following instructions should only be allowed in kernel mode, and why?
                    \begin{enumerate}[(a)]
                        \itemsep0em
                        \item Disable all interrupts \hfill kernel only
                            \medskip

                            If something in user were to disable interrupts, the kernel would have no way of regaining control,
                        \item Read the time of day clock \hfill user
                            \medskip

                            All user processes should be able to access the time if needed.
                        \item Change the memory map \hfill kernel only
                            \medskip

                            Managing memory should be restricted to the kernel.
                        \item Set the time of day \hfill kernel only
                            \medskip

                            Processes running in user space should not be able to change the time, as it can cause issues for other processes.
                    \end{enumerate}
                \item
                    A portable operating system is one that can be ported from one system architecture to another with little modification.
                    Explain why it is infeasible to build an operating system that is portable without any modification.
                    Describe two general parts that you can find in an operating system that has been designed to be highly portable.
                    \medskip

                    Since the operating system must interact with the hardware, it's not feasible to build an OS that can interact with every hardware configuration.
                    Device drivers (\textbf{platform specific}) allow for the operating system to interact with hardware, and this can be provided by the hardware manufacturer.
                    Another part would be an API the OS provides to programs (\textbf{platform independent}), allowing them to interact with hardware via this abstraction.
            \end{enumerate}
        \subsection*{Tutorial 2 - Processes + Threads}
            \begin{enumerate}[1.]
                \itemsep0em
                \item
                    If a multithreaded process forks, a problem occurs if the child gets copies of all the parent's threads.
                    Suppose that one of the original threads was waiting for keyboard input.
                    Now two threads are waiting for keyboard input, one in each process.
                    Does this problem ever occur in single-threaded processes?
                    \medskip

                    No, this does not happen in single-threaded processes as the entire process would be blocked when waiting for input, and therefore cannot fork.
                \item
                    What is the biggest advantage of implementing threads in user space?
                    What is the biggest disadvantage?
                    \medskip

                    The biggest advantage is that it allows the thread scheduling to be managed by the programmer, and also avoids the overhead of context switching.
                    On the other hand, the main disadvantage is that if any of the user space threads were to block (to wait for input), it would context switch to another thread.
                \item If in a multithreaded web server the only way to read from a file is the normal blocking ~read()~ system call, do you think user-level threads or kernel-level threads are being used?
                    \medskip

                    Kernel-level threads are being used, as the entire web server would block if it was done with user space threads.
                \item
                    Why would a thread ever voluntarily give up the CPU by calling ~thread\_yield()~?
                    After all, since there is no periodic clock interrupts, it may never get the CPU back.
                    \medskip

                    It is done to lower the priority of the thread in the scheduler, allowing another thread to run.
                    This allows threads to cooperate.
                \item
                    The register set is a per-thread rather than a per-process item.
                    Why?
                    After all, the machine has only one set of registers.
                    \medskip

                    When a thread is stopped, it has its own contents in a register which must be saved (and then restored transparently to the thread).
                \item
                    In a system with threads, is there one stack per thread or one stack per process when user-level threads are used?
                    What about when kernel threads are used?
                    Explain.
                    \medskip

                    Since each thread can call procedures, it must have a stack per thread in order to store local variables and calls.
                    The same can be said for kernel threads.
                \item
                    In this problem you are to compare reading a file using a single-threaded file server and a multithreaded server, running on a single CPU-machine.
                    It takes 15 ms to get a request for work, dispatch it, and do the rest of the necessary processing, assuming that the data needed are in the block cache.
                    If a disk operation is needed, as is the case one-third of the time, an additional 75 ms is required, during which time the thread sleeps.
                    For this problem, assume that thread switching time is negligible.
                    How many requests/sec can the server handle if it is single-threaded?
                    If it is multithreaded?
                    \medskip

                    Let the average time for an operation be $15 + \frac{1}{3} \cdot 75 = 40$ ms.
                    Therefore, with a single thread, it can process 25 requests/s.
                    \medskip

                    We want to calculate the probability that all threads are waiting for I/O.
                    The average blocking time per thread is 25 ms, as seen above.
                    This means that the probability that all threads are blocked is $1 - (\frac{25}{40})^n = 1 - (\frac{5}{8})^n$,
                    As such, we can calculate the number of requests per second as;
                    $$\left(1 - \left(\frac{5}{8}\right)^n\right) \cdot \frac{1000}{15}$$
                    Note $\frac{1000}{15}$ requests/sec is at full efficiency.
                \item
                    Would an algorithm that performs several independent CPU-intensive calculations concurrently (e.g. matrix multiplication) be more efficient if it used threads, or if it did not use threads?
                    Why is this a hard question to answer?
                    \medskip

                    It would be more efficient as long as the overhead for threads is negligible compared to the performance increase, and the CPU is actually able to perform the computation in parallel (multiple cores), otherwise the overhead of threads will cause it to be less efficient.
                    This is difficult to answer as it depends on the problem itself, how it is divided, as well as the system specifications.
                \item IPC mechanisms
                    \begin{enumerate}[(a)]
                        \itemsep0em
                        \item What happens when a signal is received by a process?
                            \medskip

                            Other than ~SIGKILL~ and ~SIGSTOP~, the receiving process is able to choose what it does with the signal.
                            It can either ignore it, or manually handle it.
                            Otherwise, it generally terminates the process.
                        \item
                            When two processes communicate through a pipe, the kernel allocates a buffer (of size 65536 bytes in Linux) for the pipe.
                            What happens when the process at the write-end of the pipe attempts to sen additional bytes on a full pipe?
                            \medskip

                            It cannot write, and will block until the process at the read-end reads from it, thus freeing up space.
                        \item
                            What happens when the process at the write-end of the pipe attempts to send additional bytes and the process at the read-end has already closed the file descriptor associated with the read-end of the pipe?
                            \medskip

                            The writing process will have an error returned to it.
                        \item
                            The process at the write-end of the pipe wants to transmit a linked list data structure (with one integer field, an a "next" pointer) over a pipe.
                            How can it do this?
                            \medskip

                            Since they do not share address spaces, it must be serialised in some form by the sending process, which can then be converted back into a linked list by the receiving process.
                        \item
                            When would it be better for two processes to communicate via shared memory instead of pipes?
                            What about the other way around?
                            \medskip

                            It would be better for two processes to communicate via shared memory as it is faster due to the lack of kernel intervention - it also allows for bi-directional communication.
                            On the other hand, pipes are handled by the kernel, thus synchronisation does not have to be implemented by the programmer.
                    \end{enumerate}
            \end{enumerate}
        \subsection*{Tutorial 3 - Scheduling}
            \begin{enumerate}[1.]
                \itemsep0em
                \item 
            \end{enumerate}
        \subsection*{Tutorial 4 - Synchronisation}
            \begin{enumerate}[1.]
                \itemsep0em
                \item 
            \end{enumerate}
        \subsection*{Tutorial 5 - Deadlocks}
            \begin{enumerate}[1.]
                \itemsep0em
                \item 
            \end{enumerate}
        \subsection*{Tutorial 6 - Memory Management}
            \begin{enumerate}[1.]
                \itemsep0em
                \item 
            \end{enumerate}
        \subsection*{Tutorial 7 - Device Management}
            \begin{enumerate}[1.]
                \itemsep0em
                \item 
            \end{enumerate}
        \subsection*{Tutorial 8 - Disk Management}
            \begin{enumerate}[1.]
                \itemsep0em
                \item 
            \end{enumerate}
        \subsection*{Tutorial 9 - File Systems}
            \begin{enumerate}[1.]
                \itemsep0em
                \item 
            \end{enumerate}
        \subsection*{Tutorial 10 - Security}
            \begin{enumerate}[1.]
                \itemsep0em
                \item 
            \end{enumerate}
\end{document}