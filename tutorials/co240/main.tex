\documentclass[a4paper, 12pt]{article}

% packages
\usepackage{amssymb}
\usepackage[fleqn]{mathtools}
\usepackage{tikz}
\usepackage{enumerate}
\usepackage{bussproofs}
\usepackage{xcolor}
\usepackage[margin=1.3cm]{geometry}
\usepackage{logicproof}
\usepackage{diagbox}
\usepackage{listings}
\usepackage{graphicx}
\usepackage{lstautogobble}
\usepackage{hyperref}
\usepackage{multirow}
\usepackage{tipa}
\usepackage{pgfplots}

% tikz libraries
\usetikzlibrary{
    decorations.pathreplacing,
    arrows,
    shapes.gates.logic.US,
    circuits.logic.US,
    calc,
    automata,
    positioning,
    intersections
}

\pgfplotsset{compat=1.16}

\pgfmathdeclarefunction{gauss}{2}{%
  \pgfmathparse{1/(#2*sqrt(2*pi))*exp(-((x-#1)^2)/(2*#2^2))}%
}

\allowdisplaybreaks % allow environments to break
\setlength\parindent{0pt} % no indent

% shorthand for verbatim
% this clashes with logicproof, so maybe fix this at some point?
\catcode`~=\active
\def~#1~{\texttt{#1}}

% code listing
\lstdefinestyle{main}{
    numberstyle=\tiny,
    breaklines=true,
    showspaces=false,
    showstringspaces=false,
    tabsize=2,
    numbers=left,
    basicstyle=\ttfamily,
    columns=fixed,
    fontadjust=true,
    basewidth=0.5em,
    autogobble,
    xleftmargin=3.0ex,
    mathescape=true
}
\newcommand{\dollar}{\mbox{\textdollar}} %
\lstset{style=main}

% augmented matrix
\makeatletter
\renewcommand*\env@matrix[1][*\c@MaxMatrixCols c]{%
\hskip -\arraycolsep
\let\@ifnextchar\new@ifnextchar
\array{#1}}
\makeatother

% ceiling / floor
\DeclarePairedDelimiter{\ceil}{\lceil}{\rceil}
\DeclarePairedDelimiter{\floor}{\lfloor}{\rfloor}

% custom commands
\newcommand{\indefint}[2]{\int #1 \, \mathrm{d}#2}
\newcommand{\defint}[4]{\int_{#1}^{#2} #3 \, \mathrm{d}#4}
\newcommand{\pdif}[2]{\frac{\partial #1}{\partial #2}}
\newcommand{\dif}[2]{\frac{\mathrm{d}#1}{\mathrm{d}#2}}
\newcommand{\limit}[2]{\raisebox{0.5ex}{\scalebox{0.8}{$\displaystyle{\lim_{#1 \to #2}}$}}}
\newcommand{\limitsup}[2]{\raisebox{0.5ex}{\scalebox{0.8}{$\displaystyle{\limsup_{#1 \to #2}}$}}}
\newcommand{\summation}[2]{\sum\limits_{#1}^{#2}}
\newcommand{\product}[2]{\prod\limits_{#1}^{#2}}
\newcommand{\intbracket}[3]{\left[#3\right]_{#1}^{#2}}
\newcommand{\laplace}{\mathcal{L}}
\newcommand{\fourier}{\mathcal{F}}
\newcommand{\mat}[1]{\boldsymbol{#1}}
\renewcommand{\vec}[1]{\boldsymbol{#1}}
\newcommand{\rowt}[1]{\begin{bmatrix}
    #1
\end{bmatrix}^\top}
\DeclareMathOperator*{\argmax}{argmax}
\DeclareMathOperator*{\argmin}{argmin}

\newcommand{\lto}[0]{\leadsto\ }

\newcommand{\ulsmash}[1]{\underline{\smash{#1}}}

\newcommand{\powerset}[0]{\wp}
\renewcommand{\emptyset}[0]{\varnothing}

\makeatletter
\newsavebox{\@brx}
\newcommand{\llangle}[1][]{\savebox{\@brx}{\(\m@th{#1\langle}\)}%
  \mathopen{\copy\@brx\kern-0.5\wd\@brx\usebox{\@brx}}}
\newcommand{\rrangle}[1][]{\savebox{\@brx}{\(\m@th{#1\rangle}\)}%
  \mathclose{\copy\@brx\kern-0.5\wd\@brx\usebox{\@brx}}}
\makeatother
\newcommand{\lla}{\llangle}
\newcommand{\rra}{\rrangle}
\newcommand{\la}{\langle}
\newcommand{\ra}{\rangle}
\newcommand{\crnr}[1]{\text{\textopencorner} #1 \text{\textcorner}}
\newcommand{\bnfsep}[0]{\ |\ }
\newcommand{\concsep}[0]{\ ||\ }

\newcommand{\axiom}[1]{\AxiomC{#1}}
\newcommand{\unary}[1]{\UnaryInfC{#1}}
\newcommand{\binary}[1]{\BinaryInfC{#1}}
\newcommand{\trinary}[1]{\TrinaryInfC{#1}}
\newcommand{\quaternary}[1]{\QuaternaryInfC{#1}}
\newcommand{\quinary}[1]{\QuinaryInfC{#1}}
\newcommand{\dproof}[0]{\DisplayProof}
\newcommand{\llabel}[1]{\LeftLabel{\scriptsize #1}}
\newcommand{\rlabel}[1]{\RightLabel{\scriptsize #1}}

\newcommand{\ttbs}{\char`\\}
\newcommand{\lrbt}[0]{\ \bullet\ }

% colours
\newcommand{\violet}[1]{\textcolor{violet}{#1}}
\newcommand{\blue}[1]{\textcolor{blue}{#1}}
\newcommand{\red}[1]{\textcolor{red}{#1}}
\newcommand{\teal}[1]{\textcolor{teal}{#1}}

% reasoning proofs
\usepackage{ltablex}
\usepackage{environ}
\keepXColumns
\NewEnviron{reasoning}{
    \begin{tabularx}{\textwidth}{rlX}
        \BODY
    \end{tabularx}
}
\newcommand{\proofline}[3]{$(#1)$ & $#2$ & \hfill #3 \smallskip \\}
\newcommand{\proofarbitrary}[1]{& take arbitrary $#1$ \smallskip \\}
\newcommand{\prooftext}[1]{\multicolumn{3}{l}{#1} \smallskip \\}
\newcommand{\proofmath}[3]{$#1$ & = $#2$ & \hfill #3 \smallskip \\}
\newcommand{\prooftherefore}[1]{& $\therefore #1$ \smallskip \\}
\newcommand{\proofbc}[0]{\prooftext{\textbf{Base Case}}}
\newcommand{\proofis}[0]{\prooftext{\textbf{Inductive Step}}}

% ER diagrams
\newcommand{\nattribute}[4]{
    \node[draw, state, inner sep=0cm, minimum size=0.2cm, label=#3:{#4}] (#1) at (#2) {};
}
\newcommand{\mattribute}[4]{
    \node[draw, state, accepting, inner sep=0cm, minimum size=0.2cm, label=#3:{#4}] (#1) at (#2) {};
}
\newcommand{\dattribute}[4]{
    \node[draw, state, dashed, inner sep=0cm, minimum size=0.2cm, label=#3:{#4}] (#1) at (#2) {};
}
\newcommand{\entity}[3]{
    \node[] (#1-c) at (#2) {#3};
    \node[inner sep=0cm] (#1-l) at ($(#1-c) + (-1, 0)$) {};
    \node[inner sep=0cm] (#1-r) at ($(#1-c) + (1, 0)$) {};
    \node[inner sep=0cm] (#1-u) at ($(#1-c) + (0, 0.5)$) {};
    \node[inner sep=0cm] (#1-d) at ($(#1-c) + (0, -0.5)$) {};
    \draw
    ($(#1-c) + (-1, 0.5)$) -- ($(#1-c) + (1, 0.5)$) -- ($(#1-c) + (1, -0.5)$) -- ($(#1-c) + (-1, -0.5)$) -- cycle;
}
\newcommand{\relationship}[3]{
    \node[] (#1-c) at (#2) {#3};
    \node[inner sep=0cm] (#1-l) at ($(#1-c) + (-1, 0)$) {};
    \node[inner sep=0cm] (#1-r) at ($(#1-c) + (1, 0)$) {};
    \node[inner sep=0cm] (#1-u) at ($(#1-c) + (0, 1)$) {};
    \node[inner sep=0cm] (#1-d) at ($(#1-c) + (0, -1)$) {};
    \draw
    ($(#1-c) + (-1, 0)$) -- ($(#1-c) + (0, 1)$) -- ($(#1-c) + (1, 0)$) -- ($(#1-c) + (0, -1)$) -- cycle;
}

% AVL Trees
\newcommand{\avltri}[4]{
    \draw ($(#1)$) -- ($(#1) + #4*(0.5, -1)$) -- ($(#1) + #4*(-0.5, -1)$) -- cycle;
    \node at ($(#1) + #4*(0, -1) + (0, 0.5)$) {#3};
    \node at ($(#1) + #4*(0, -1) + (0, -0.5)$) {#2};
}

% RB Trees
\tikzset{rbtr/.style={inner sep=2pt, circle, draw=black, fill=red}}
\tikzset{rbtb/.style={inner sep=2pt, circle, draw=black, fill=black}}

% actual document
\begin{document}
    \section*{CO240 - Models of Computation \hfill Tutorial Sheets}
        \subsection*{Tutorial 1 - Expressions}
            \begin{enumerate}[1.]
                \itemsep0em
                \item
                    Consider the \textbf{big-step} operational semantics for the language \textit{SimpleExp} given in the lectures.
                    Find a number $n$ such that
                    \begin{center}
                        $(4 + 1) + (2 + 2) \Downarrow n$
                    \end{center}
                    Give the full derivation tree.
                    \begin{center}
                                    \axiom{}
                                \llabel{(B-NUM)}
                                \unary{$4 \Downarrow 4$}
                                    \axiom{}
                                \llabel{(B-NUM)}
                                \unary{$1 \Downarrow 1$}
                            \llabel{(B-ADD)}
                            \binary{$(4 + 1) \Downarrow 5$}
                                    \axiom{}
                                \llabel{(B-NUM)}
                                \unary{$2 \Downarrow 2$}
                                    \axiom{}
                                \llabel{(B-NUM)}
                                \unary{$2 \Downarrow 2$}
                            \llabel{(B-ADD)}
                            \binary{$(2 + 2) \Downarrow 2$}
                        \llabel{(B-ADD)}
                        \binary{$(4 + 1) + (2 + 2) \Downarrow 9$}
                        \dproof
                    \end{center}
                \item
                    The big-step operation semantics for \textit{SimpleExp} was only given for addition.
                    Extend it to include \textit{multiplication}.
                    Give a proof that $((3 + 2) \times (1 + 4)) \Downarrow 25$
                    \medskip

                    To do this, we need to add an additional rule as follows;
                    \begin{center}
                            \axiom{$E_1 \Downarrow n_1$}
                            \axiom{$E_2 \Downarrow n_2$}
                        \llabel{(B-MUL)}
                        \rlabel{$n_3 = n_1 \times n_2$}
                        \binary{$E_1 \times E_2 \Downarrow n_3$}
                        \dproof
                    \end{center}
                    Hence we can do the following;
                    \begin{center}
                                    \axiom{}
                                \llabel{(B-NUM)}
                                \unary{$3 \Downarrow 3$}
                                    \axiom{}
                                \llabel{(B-NUM)}
                                \unary{$2 \Downarrow 2$}
                            \llabel{(B-ADD)}
                            \binary{$(3 + 2) \Downarrow 5$}
                                    \axiom{}
                                \llabel{(B-NUM)}
                                \unary{$1 \Downarrow 1$}
                                    \axiom{}
                                \llabel{(B-NUM)}
                                \unary{$4 \Downarrow 4$}
                            \llabel{(B-ADD)}
                            \binary{$(1 + 4) \Downarrow 5$}
                        \llabel{(B-MUL)}
                        \binary{$((3 + 2) \times (1 + 4)) \Downarrow 25$}
                        \dproof
                    \end{center}
                \item
                    Extend the \textbf{big-step} semantics further to include \textit{subtraction}.
                    Remember that the numbers in the syntax of the language are $0, 1, 2, \dots$ (no negative numbers).
                    \smallskip

                    How is an expression such as $(3 - 7)$ handled in your semantics?
                    Have you made any arbitrary decisions about this?
                    If so, what other options were available?
                    \medskip

                    Note that this question has multiple valid options; we can either introduce a ~NaN~ concept, representing an "invalid" operation, which has to be propagated in all rules, or we could have it be some value.
                    The latter can lead to ambiguity, because if we had $(3 - 7) \Downarrow 0$, and also $(4 - 7) \Downarrow 0$, we may unexpected results.
                \item Recall the \textbf{small-step} operational semantics of \textit{SimpleExp}.
                    \begin{enumerate}[(a)]
                        \itemsep0em
                        \item
                            Give the full derivation of the first step of evaluation of $((1 + 2) + (4 + 3))$ - give the derivation tree of the step (for some expression $E$);
                            \begin{center}
                                $((1 + 2) + (4 + 3)) \to E$
                            \end{center}
                            For the first step, we have the following;
                            \begin{center}
                                        \axiom{}
                                    \llabel{(S-ADD)}
                                    \unary{$(1 + 2) \to 3$}
                                \llabel{(S-LEFT)}
                                \unary{$((1 + 2) + (4 + 3)) \to (3 + (4 + 3))$}
                                \dproof
                            \end{center}
                        \item
                            Write down all the steps of evaluation needed to reduce the above expression to 10.
                            Give the full derivation for each of these steps.
                            \medskip

                            Note that the \textbf{evaluation path} is;
                            \begin{center}
                                $((1 + 2) + (4 + 3)) \to (3 + (4 + 3)) \to (3 + 7) \to 10$
                            \end{center}
                            The derivation tree for each step is as follows;
                            \begin{center}
                                        \axiom{}
                                    \llabel{(S-ADD)}
                                    \unary{$(4 + 3) \to 7$}
                                \llabel{(S-RIGHT)}
                                \unary{$(3 + (4 + 3)) \to (3 + 7)$}
                                \dproof
                            \end{center}
                            Followed by;
                            \begin{center}
                                    \axiom{}
                                \llabel{(S-ADD)}
                                \unary{$(3 + 7) \to 10$}
                                \dproof
                            \end{center}
                    \end{enumerate}
                \item Here is the abstract syntax for a simple language \textit{Bool} of boolean expressions:
                    \begin{center}
                        $B \in \textit{Bool} ::= ~true~ \bnfsep ~false~ \bnfsep B \& B \bnfsep \neg B \bnfsep ~if ~ B ~ then ~ B ~ else ~ B$
                    \end{center}
                    Intuitively, every expression evaluates to either ~true~ or ~false~.
                    \begin{enumerate}[(a)]
                        \itemsep0em
                        \item Give a \textbf{small-step} operational semantics for \textit{Bool}.
                            \begin{center}
                                    \axiom{$B_1 \to B_1^\prime$}
                                \unary{$B_1 \& B_2 \to B_1^\prime \& B_2$}
                                \dproof
                                \medskip

                                    \axiom{$B_2 \to B_2^\prime$}
                                \unary{$~true~ \& B_2 \to ~true~ \& B_2^\prime$}
                                \dproof
                                \medskip

                                    \axiom{$B_2 \to B_2^\prime$}
                                \unary{$~false~ \& B_2 \to ~false~ \& B_2^\prime$}
                                \dproof
                                \medskip

                                    \axiom{}
                                \unary{$~true~ \& ~true~ \to ~true~$}
                                \dproof
                                \medskip

                                    \axiom{}
                                \unary{$~true~ \& ~false~ \to ~false~$}
                                \dproof
                                \medskip

                                    \axiom{}
                                \unary{$~false~ \& ~true~ \to ~false~$}
                                \dproof
                                \medskip

                                    \axiom{}
                                \unary{$~false~ \& ~false~ \to ~false~$}
                                \dproof
                                \medskip

                                    \axiom{$B \to B^\prime$}
                                \unary{$\neg B \to \neg B^\prime$}
                                \dproof
                                \medskip

                                    \axiom{}
                                \unary{$\neg ~true~ \to ~false~$}
                                \dproof
                                \medskip

                                    \axiom{}
                                \unary{$\neg ~false~ \to ~true~$}
                                \dproof
                                \medskip

                                    \axiom{$B_1 \to B_1^\prime$}
                                \unary{$~if ~ B_1 ~ then ~ B_2 ~ else ~ B_3$}
                                \dproof
                                \medskip

                                    \axiom{}
                                \unary{$~if true then ~ B_2 ~ else ~ B_3 \to B_2$}
                                \dproof
                                \medskip

                                    \axiom{}
                                \unary{$~if false then ~ B_2 ~ else ~ B_3 \to B_3$}
                                \dproof
                            \end{center}
                            Note that these are all evaluated right-to-left.
                        \item Write down all the steps of evaluation needed to reduce the following expression to a result:
                            \begin{align*}
                                & \neg (~if ~ (~false~ \& ~true~) ~ then ~ (~if true then ~ (~false~ \& ~true~) ~ else false~) ~ else ~ \neg ~true~) \\
                                \to\ & \neg (~if false then ~ (~if true then ~ (~false~ \& ~true~) ~ else false~) ~ else ~ \neg ~true~) \\
                                \to\ & \neg (\neg ~true~) \\
                                \to\ & \neg ~false~ \\
                                \to\ & ~true~
                            \end{align*}
                    \end{enumerate}
                \item
                    The syntax of \textit{SimpleExp} is extended with a new operator $?$, as follows;
                    \begin{center}
                        $E \in \textit{SimpleExp} ::= \dots \bnfsep (E ? E)$
                    \end{center}
                    This operator allows the implementation to choose to give the result of $E_1$, or $E_2$, when given $E_1 ? E_2$.
                    \begin{enumerate}[(a)]
                        \itemsep0em
                        \item Extend the \textbf{big-step} operational semantics with rules for $?$ that capture this meaning.
                            \begin{center}
                                \hfill
                                    \axiom{$E_1 \Downarrow n_1$}
                                \llabel{(B-CHOICE-1)}
                                \unary{$E_1 ? E_2 \Downarrow n_1$}
                                \dproof
                                \hfill
                                    \axiom{$E_2 \Downarrow n_2$}
                                \llabel{(B-CHOICE-2)}
                                \unary{$E_1 ? E_2 \Downarrow n_2$}
                                \dproof
                                \hfill \phantom{}
                            \end{center}
                        \item For what values of $n$ does $(0 ? 1) + (2 ? 3) \Downarrow n$?
                            \begin{center}
                                            \axiom{}
                                        \llabel{(B-NUM)}
                                        \unary{$0 \Downarrow 0$}
                                    \llabel{(B-CHOICE-1)}
                                    \unary{$(0 ? 1) \Downarrow 0$}
                                            \axiom{}
                                        \llabel{(B-NUM)}
                                        \unary{$2 \Downarrow 2$}
                                    \llabel{(B-CHOICE-1)}
                                    \unary{$(2 ? 3) \Downarrow 2$}
                                \llabel{(B-ADD)}
                                \binary{$(0 ? 1) + (2 ? 3) \Downarrow 2$}
                                \dproof
                                \medskip

                                            \axiom{}
                                        \llabel{(B-NUM)}
                                        \unary{$0 \Downarrow 0$}
                                    \llabel{(B-CHOICE-1)}
                                    \unary{$(0 ? 1) \Downarrow 0$}
                                            \axiom{}
                                        \llabel{(B-NUM)}
                                        \unary{$3 \Downarrow 3$}
                                    \llabel{(B-CHOICE-2)}
                                    \unary{$(2 ? 3) \Downarrow 3$}
                                \llabel{(B-ADD)}
                                \binary{$(0 ? 1) + (2 ? 3) \Downarrow 3$}
                                \dproof
                                \medskip

                                            \axiom{}
                                        \llabel{(B-NUM)}
                                        \unary{$1 \Downarrow 1$}
                                    \llabel{(B-CHOICE-2)}
                                    \unary{$(0 ? 1) \Downarrow 1$}
                                            \axiom{}
                                        \llabel{(B-NUM)}
                                        \unary{$2 \Downarrow 2$}
                                    \llabel{(B-CHOICE-1)}
                                    \unary{$(2 ? 3) \Downarrow 2$}
                                \llabel{(B-ADD)}
                                \binary{$(0 ? 1) + (2 ? 3) \Downarrow 3$}
                                \dproof
                                \medskip

                                            \axiom{}
                                        \llabel{(B-NUM)}
                                        \unary{$1 \Downarrow 1$}
                                    \llabel{(B-CHOICE-2)}
                                    \unary{$(0 ? 1) \Downarrow 1$}
                                            \axiom{}
                                        \llabel{(B-NUM)}
                                        \unary{$3 \Downarrow 3$}
                                    \llabel{(B-CHOICE-2)}
                                    \unary{$(2 ? 3) \Downarrow 3$}
                                \llabel{(B-ADD)}
                                \binary{$(0 ? 1) + (2 ? 3) \Downarrow 4$}
                                \dproof
                            \end{center}
                        \item
                            Is the semantics deterministic?
                            Is it total?
                            \medskip

                            It is not deterministic as we have $0 ? 1 \Downarrow 0$, as well as $0 ? 1 \Downarrow 1$ - but $0 \neq 1$.
                            It is total as it is applies to every expression (for something to be total, we need some number $n$ for every expression $E$ such that $E \Downarrow n$).
                    \end{enumerate}
                \item
                    \begin{enumerate}[(a)]
                        \itemsep0em
                        \item Extend the \textbf{small-step} semantics for \textit{SimpleExp} to handle the $?$ operator by adding appropriate derivation rules for $\to$.
                            \begin{center}
                                \hfill
                                    \axiom{}
                                \llabel{(S-CHOICE-1)}
                                \unary{$E_1 ? E_2 \to E_1$}
                                \dproof
                                \hfill
                                    \axiom{}
                                \llabel{(S-CHOICE-2)}
                                \unary{$E_1 ? E_2 \to E_2$}
                                \dproof
                                \hfill \phantom{}
                            \end{center}
                        \item Give all possible derivations of the first step of evaluation of $(0 ? 1) + (2 ? 3)$.
                            \begin{center}
                                \hfill
                                        \axiom{}
                                    \llabel{(S-CHOICE-1)}
                                    \unary{$0 ? 1 \to 0$}
                                \llabel{(S-LEFT)}
                                \unary{$(0 ? 1) + (2 ? 3) \to 0 + (2 ? 3)$}
                                \dproof
                                \hfill
                                        \axiom{}
                                    \llabel{(S-CHOICE-2)}
                                    \unary{$0 ? 1 \to 1$}
                                \llabel{(S-LEFT)}
                                \unary{$(0 ? 1) + (2 ? 3) \to 1 + (2 ? 3)$}
                                \dproof
                                \hfill \phantom{}
                            \end{center}
                        \item Give all of the possible evaluation paths for $(0 ? 1) + (2 ? 3)$.
                            \begin{center}
                                $(0 ? 1) + (2 ? 3) \to 0 + (2 ? 3) \to 0 + 2 \to 2$

                                $(0 ? 1) + (2 ? 3) \to 0 + (2 ? 3) \to 0 + 3 \to 3$

                                $(0 ? 1) + (2 ? 3) \to 1 + (2 ? 3) \to 1 + 2 \to 3$

                                $(0 ? 1) + (2 ? 3) \to 1 + (2 ? 3) \to 1 + 3 \to 4$
                            \end{center}
                        \item Is the semantics confluent?
                            \medskip

                            We've shown $(0 ? 1) + (2 ? 3) \to^* 2$ and also $(0 ? 1) + (2 ? 3) \to^* 3$.
                            Therefore, for the semantics to be confluent, there must be some $E^\prime$ such that $2 \to^* E^\prime$ and $3 \to^* E^\prime$ - however, since they are both in normal forms, they can only evaluate to themselves.
                            $2 \neq 3$, hence it is not confluent.
                        \item Is the semantics normalising?
                            \medskip

                            Yes, there are no infinite sequences of expressions, hence any evaluation path will eventually reach a normal form.
                    \end{enumerate}
                \item Suppose that instead of the \textit{SimpleExp} small-step rule {\scriptsize (S-RIGHT)}, we had the following;
                    \begin{center}
                            \axiom{$E_2 \to E_2^\prime$}
                        \llabel{(S-RIGHT$^\prime$)}
                        \unary{$(E_1 + E_2) \to (E_1 + E_2^\prime)$}
                        \dproof
                    \end{center}
                    \begin{enumerate}[(a)]
                        \itemsep0em
                        \item Given an evaluation path using the {\scriptsize S-RIGHT} rule, is it also an evaluation path using the {\scriptsize S-RIGHT$^\prime$} rule?
                            \medskip

                            Yes, as the original rule constrained $E_1$ to be in a normal form, but the new rule doesn't.
                            This means that the new rule covers all the cases of the original rule.
                        \item Find an expression that has an evaluation path using the {\scriptsize S-RIGHT$^\prime$} rule that it did not have with the {\scriptsize S-RIGHT} rule.
                            \begin{center}
                                $(0 + 1) + (2 + 3) \to (0 + 1) + 5 \to 1 + 5 \to 6$
                            \end{center}
                        \item Is $\to$ deterministic?
                            \medskip

                            No, starting with $(0 + 1) + (2 + 3)$, we can go to either $1 + (2 + 3)$ {\scriptsize S-LEFT}, or $(0 + 1) + 5$ with {\scriptsize S-RIGHT$^\prime$} - however the two expressions are not equal.
                        \item Is $\to$ confluent?
                            \medskip

                            Yes, the rule allows for different evaluation order, but doesn't change the result of the evaluation.
                    \end{enumerate}
            \end{enumerate}
        \subsection*{Tutorial 2 - State}
            \begin{enumerate}[1.]
                \itemsep0em
                \item
                    Consider the small-step operation semantics of the language \textit{While}.
                    Write down all of the evaluation steps of the program $(z := x; x := y); y := z$, with the initial state $s = (x \mapsto 5, y \mapsto 7)$.
                    Give the full derivation tree for the first step in this evaluation.
                    \begin{center}
                                        \axiom{}
                                    \llabel{(W-EXP.VAR)}
                                    \unary{$\la x, (x \mapsto 5, y \mapsto 7) \ra \to_e \la 5, (x \mapsto 5, y \mapsto 7) \ra$}
                                \llabel{(W-ASS.EXP)}
                                \unary{$\la z := x, (x \mapsto 5, y \mapsto 7) \ra \to_c \la z := 5, (x \mapsto 5, y \mapsto 7) \ra$}
                            \llabel{(W-SEQ.LEFT)}
                            \unary{$\la z := x; x := y, (x \mapsto 5, y \mapsto 7) \ra \to_c \la z := 5; x := y, (x \mapsto 5, y \mapsto 7) \ra$}
                        \llabel{(W-SEQ.LEFT)}
                        \unary{$\la (z := x; x := y); y := z, (x \mapsto 5, y \mapsto 7) \ra \to_c \la (z := 5; x := y); y := z, (x \mapsto 5, y \mapsto 7) \ra$}
                        \dproof
                    \end{center}
                    All of the steps are as follows;
                    \begin{align*}
                        & \la (z := x; x := y); y := z, (x \mapsto 5, y \mapsto 7) \ra \\
                        \to_c\ & \la (z := 5; x := y); y := z, (x \mapsto 5, y \mapsto 7) \ra \\
                        \to_c\ & \la (~skip~; x := y); y := z, (x \mapsto 5, y \mapsto 7, z \mapsto 5) \ra \\
                        \to_c\ & \la x := y; y := z, (x \mapsto 5, y \mapsto 7, z \mapsto 5) \ra \\
                        \to_c\ & \la x := 7; y := z, (x \mapsto 5, y \mapsto 7, z \mapsto 5) \ra \\
                        \to_c\ & \la ~skip~; y := z, (x \mapsto 7, y \mapsto 7, z \mapsto 5) \ra \\
                        \to_c\ & \la y := z, (x \mapsto 7, y \mapsto 7, z \mapsto 5) \ra \\
                        \to_c\ & \la y := 5, (x \mapsto 7, y \mapsto 7, z \mapsto 5) \ra \\
                        \to_c\ & \la ~skip~, (x \mapsto 7, y \mapsto 5, z \mapsto 5) \ra
                    \end{align*}
                \item
                    Consider the small-step operational semantics of the language \textit{While}.
                    Write down all of the evaluation steps of the program (given the initial state $s = (x \mapsto 1)$)
                    \begin{center}
                        $\violet{(\text{let } W =)}\ ~while ~ x < 4 ~ do ~ x := x + 2$
                    \end{center}
                    Give full derivation trees for the first four steps.
                    \begin{center}
                            \axiom{}
                        \llabel{1}
                        \unary{$\la ~while ~ x < 4 ~ do ~ x := x + 2, (x \mapsto 1) \ra \to_c\ \la ~if ~ x < 4 ~ then ~ (x := x + 2; W) ~ else skip~, (x \mapsto 1) \ra$}
                        \dproof
                        \medskip

                                    \axiom{}
                                \llabel{(W-EXP.VAR)}
                                \unary{$\la x, (x \mapsto 1) \ra \to_e \la 1, (x \mapsto 1) \ra$}
                            \llabel{(W-BEXP.LEFT)}
                            \unary{$\la x < 4, (x \mapsto 1) \ra \to_b \la 1 < 4, (x \mapsto 1) \ra$}
                        \llabel{2}
                        \unary{$\la ~while ~ x < 4 ~ do ~ x := x + 2, (x \mapsto 1) \ra \to_c\ \la ~if ~ 1 < 4 ~ then ~ (x := x + 2; W) ~ else skip~, (x \mapsto 1) \ra$}
                        \dproof
                        \medskip

                                \axiom{}
                            \llabel{(W-BEXP.LT)}
                            \unary{$\la 1 < 4, (x \mapsto 1) \ra \to_b \la ~true~, (x \mapsto 1) \ra$}
                        \llabel{2}
                        \unary{$\la ~while ~ x < 4 ~ do ~ x := x + 2, (x \mapsto 1) \ra \to_c\ \la ~if true then ~ (x := x + 2; W) ~ else skip~, (x \mapsto 1) \ra$}
                        \dproof
                        \medskip

                            \axiom{}
                        \llabel{(W-COND.TRUE)}
                        \unary{$\la ~if true then ~ (x := x + 2; W) ~ else skip~, (x \mapsto 1) \ra \to_c\ \la x := x + 2; W, (x \mapsto 1) \ra$}
                        \dproof
                    \end{center}
                    Note that rule 1 is {\scriptsize (W-WHILE)}, and rule 2 is {\scriptsize (W-COND.BEXP)}.
                    The full evaluation path is as follows;
                    \begin{align*}
                        & \la ~while ~ x < 4 ~ do ~ x := x + 2, (x \mapsto 1) \ra \\
                        \to_c\ & \la ~if ~ x < 4 ~ then ~ (x := x + 2; W) ~ else skip~, (x \mapsto 1) \ra \\
                        \to_c\ & \la ~if ~ 1 < 4 ~ then ~ (x := x + 2; W) ~ else skip~, (x \mapsto 1) \ra \\
                        \to_c\ & \la ~if true then ~ (x := x + 2; W) ~ else skip~, (x \mapsto 1) \ra \\
                        \to_c\ & \la x := x + 2; W, (x \mapsto 1) \ra \\
                        \to_c\ & \la x := 1 + 2; W, (x \mapsto 1) \ra \\
                        \to_c\ & \la x := 3; W, (x \mapsto 1) \ra \\
                        \to_c\ & \la ~skip~; W, (x \mapsto 3) \ra \\
                        \to_c\ & \la ~while ~ x < 4 ~ do ~ x := x + 2, (x \mapsto 3) \ra \\
                        \to_c\ & \la ~if ~ x < 4 ~ then ~ (x := x + 2; W) ~ else skip~, (x \mapsto 3) \ra \\
                        \to_c\ & \la ~if ~ 3 < 4 ~ then ~ (x := x + 2; W) ~ else skip~, (x \mapsto 3) \ra \\
                        \to_c\ & \la ~if true then ~ (x := x + 2; W) ~ else skip~, (x \mapsto 3) \ra \\
                        \to_c\ & \la x := x + 2; W, (x \mapsto 3) \ra \\
                        \to_c\ & \la x := 3 + 2; W, (x \mapsto 3) \ra \\
                        \to_c\ & \la x := 5; W, (x \mapsto 3) \ra \\
                        \to_c\ & \la ~skip~; W, (x \mapsto 5) \ra \\
                        \to_c\ & \la ~while ~ x < 4 ~ do ~ x := x + 2, (x \mapsto 5) \ra \\
                        \to_c\ & \la ~if ~ x < 4 ~ then ~ (x := x + 2; W) ~ else skip~, (x \mapsto 5) \ra \\
                        \to_c\ & \la ~if ~ 5 < 4 ~ then ~ (x := x + 2; W) ~ else skip~, (x \mapsto 5) \ra \\
                        \to_c\ & \la ~if false then ~ (x := x + 2; W) ~ else skip~, (x \mapsto 5) \ra \\
                        \to_c\ & \la ~skip~, (x \mapsto 5) \ra
                    \end{align*}
                \item
                    Consider adding the increment expression $x++$ to the language \textit{While}.
                    The expression returns the value of the variable (only applied to variables) $x$ and then updates the value of $x$ to be one greater than the old value; its semantics is given by the following rule:
                    \begin{center}
                            \axiom{}
                        \llabel{(W-EXP.PP)}
                        \rlabel{$s(x) = n, n^\prime = n + 1$}
                        \unary{$\la x++, s \ra \to_e \la n, s[x \mapsto n^\prime] \ra$}
                        \dproof
                    \end{center}
                    \begin{enumerate}[(a)]
                        \itemsep0em
                        \item Give the full execution path for the program $x := (x++) + (x++)$ from the initial state $(x \mapsto 2)$.
                            \begin{align*}
                                & \la x := (x++) + (x++), (x \mapsto 2) \ra \\
                                \to_c\ & \la x := 2 + (x++), (x \mapsto 3) \ra \\
                                \to_c\ & \la x := 2 + 3, (x \mapsto 4) \ra \\
                                \to_c\ & \la x := 5, (x \mapsto 4) \ra \\
                                \to_c\ & \la ~skip~, (x \mapsto 5) \ra
                            \end{align*}
                        \item Given an operational semantics rule for $++x$, which increments $x$ and then returns the result.
                            \begin{center}
                                    \axiom{}
                                \llabel{(W-EXP.PP)}
                                \rlabel{$s(x) = n, n^\prime = n + 1$}
                                \unary{$\la ++x, s \ra \to_e \la n^\prime, s[x \mapsto n^\prime] \ra$}
                                \dproof
                            \end{center}
                    \end{enumerate}
                \item
                    Consider what happens if we add a 'side-effecting expression' of the form
                    \begin{center}
                        $~do ~ C ~ return ~ E$
                    \end{center}
                    This runs first runs the command $C$, and returns the value of $E$.
                    \begin{center}
                        \hfill
                            \axiom{$\la C, s \ra \to_c \la C^\prime, s^\prime \ra$}
                        \unary{$\la ~do ~ C ~ return ~ E, s \ra \to_e \la ~do ~ C^\prime ~ return ~ E, s^\prime \ra$}
                        \dproof
                        \hfill
                            \axiom{}
                        \unary{$\la ~do skip return ~ E, s \ra \to_e \la E, s \ra$}
                        \dproof
                        \hfill \phantom{}
                    \end{center}
                \item
                    Consider the \textit{While} language extend with parallel composition of commands: $C \concsep C$.
                    The semantics of parallel composition is given by interleaving the execution steps of the two composed commands in an arbitrary fashion.
                    This is expressed formally as;
                    \begin{center}
                        \hfill
                            \axiom{$\la C_1, s \ra \to_c \la C_1^\prime, s^\prime \ra$}
                        \unary{$\la C_1 \concsep C_2, s \ra \to_c \la C_1^\prime \concsep C_2, s^\prime \ra$}
                        \dproof
                        \hfill
                            \axiom{$\la C_2, s \ra \to_c \la C_2^\prime, s^\prime \ra$}
                        \unary{$\la C_1 \concsep C_2, s \ra \to_c \la C_1 \concsep C_2^\prime, s^\prime \ra$}
                        \dproof
                        \hfill
                            \axiom{}
                        \unary{$\la ~skip~ \concsep ~skip~, s \ra \to_c \la ~skip~, s \ra$}
                        \dproof
                        \hfill \phantom{}
                    \end{center}
                    \begin{enumerate}[(a)]
                        \itemsep0em
                        \item
                            Consider the command $(x := 1) \concsep (x := 2; x := (x + 2))$, run with initial state $s = (x \mapsto 0)$.
                            How many possible final values for $x$ does this command have?
                            \medskip

                            There are 3 possible values; 1, 3, or 4.
                        \item How many different evaluation paths exist for obtaining the final value 4?
                            \medskip

                            3 paths.
                            I really can't be bothered to type out all of the steps.
                            The point is the operation $x := x + 2$ is not atomic; even if we have obtained $x := 4$, we can execute $x := 1$, and then still obtain a state with $x \mapsto 4$, if the former is executed at the end.
                        \item
                            A useful operation in concurrency is atomic compare-and-swap.
                            This operation is added to the \textit{While} language in the form of a new boolean expression $~CAS~(x, E, E)$.
                            To execute the operation $~CAS~(x, E_1, E_2)$, first $E_1$ and then $E_2$ are evaluated to numbers $n_1$ and $n_2$ in the usual way.
                            Then, \textbf{in a single step}, the operation compares the value of variable $x$ with $n_1$; if the values are equal, it updates the value of $x$ to be number $n_2$ and returns ~true~, otherwise, it simply returns ~false~.
                            Extend the operational semantics with rules for ~CAS~ that implement this behaviour.
                            \begin{center}
                                    \axiom{$\la E_1, s \ra \to_e \la E_1^\prime, s^\prime \ra$}
                                \unary{$\la ~CAS~(x, E_1, E_2), s \ra \to_b \la ~CAS~(x, E_1^\prime, E_2), s^\prime \ra$}
                                \dproof
                                \medskip

                                    \axiom{$\la E_2, s \ra \to_e \la E_2^\prime, s^\prime \ra$}
                                \unary{$\la ~CAS~(x, n_1, E_2), s \ra \to_b \la ~CAS~(x, n_1, E_2^\prime), s^\prime \ra$}
                                \dproof
                                \medskip

                                    \axiom{}
                                \rlabel{$s(x) = n_1$}
                                \unary{$\la ~CAS~(x, n_1, n_2), s \ra \to_b \la ~true~, s[x \mapsto n_2] \ra$}
                                \dproof
                                \medskip

                                    \axiom{}
                                \rlabel{$s(x) \neq n_1$}
                                \unary{$\la ~CAS~(x, n_1, n_2), s \ra \to_b \la ~false~, s \ra$}
                                \dproof
                            \end{center}
                    \end{enumerate}
                \item
                    Suppose that $\la C_1; C_2, s \ra \to_c^* \la C_2, s^\prime \ra$.
                    Show that it is not necessarily the case that $\la C_1, s \ra \to_c^* \la ~skip~, s^\prime \ra$.
                    \medskip

                    Let there be a state $s^{\prime\prime} \neq s^\prime$, where $\la C_1, s \ra \to_c^* \la ~skip~, s^{\prime\prime} \ra$.
                    For $\la C_1; C_2, s \ra \to_c^* \la C_2, s^\prime \ra$, we can find $C_2$ such that $\la C_2, s^{\prime\prime} \ra \to_c^* \la C_2, s^\prime \ra$.
                    From here, we see that our foal is to find $C_2$ as something that evaluates to itself, but in a different state (hence a loop).
                    \begin{align*}
                        C_1 & = ~skip~ \\
                        C_2 & = ~while true do ~ x := 1 \\
                        s & = (x \mapsto 0)
                    \end{align*}
                    Executing this we have;
                    \begin{align*}
                        & \la ~while true do ~ x := 1, (x \mapsto 0) \ra \\
                        \to_c\ & \la ~if true then ~ x := 1; C_2 ~ else skip~, (x \mapsto 0) \ra \\
                        \to_c\ & \la x := 1; C_2, (x \mapsto 0) \ra \\
                        \to_c\ & \la ~skip~; C_2, (x \mapsto 1) \ra \\
                        \to_c\ & \la C_2, (x \mapsto 1) \ra \\
                    \end{align*}
            \end{enumerate}
\end{document}