\documentclass[a4paper, 12pt]{article}

% packages
\usepackage{amssymb}
\usepackage[fleqn]{mathtools}
\usepackage{tikz}
\usepackage{enumerate}
\usepackage{bussproofs}
\usepackage{xcolor}
\usepackage[margin=1.3cm]{geometry}
\usepackage{logicproof}
\usepackage{diagbox}
\usepackage{listings}
\usepackage{graphicx}
\usepackage{lstautogobble}
\usepackage{hyperref}
\usepackage{multirow}
\usepackage{tipa}
\usepackage{pgfplots}
\usepackage{adjustbox}

% tikz libraries
\usetikzlibrary{
    decorations.pathreplacing,
    arrows,
    shapes.gates.logic.US,
    circuits.logic.US,
    calc,
    automata,
    positioning,
    intersections
}

\pgfplotsset{compat=1.16}

\pgfmathdeclarefunction{gauss}{2}{%
  \pgfmathparse{1/(#2*sqrt(2*pi))*exp(-((x-#1)^2)/(2*#2^2))}%
}

\allowdisplaybreaks % allow environments to break
\setlength\parindent{0pt} % no indent

% shorthand for verbatim
% this clashes with logicproof, so maybe fix this at some point?
\catcode`~=\active
\def~#1~{\texttt{#1}}

% code listing
\lstdefinestyle{main}{
    numberstyle=\tiny,
    breaklines=true,
    showspaces=false,
    showstringspaces=false,
    tabsize=2,
    numbers=left,
    basicstyle=\ttfamily,
    columns=fixed,
    fontadjust=true,
    basewidth=0.5em,
    autogobble,
    xleftmargin=3.0ex,
    mathescape=true
}
\newcommand{\dollar}{\mbox{\textdollar}} %
\lstset{style=main}

% augmented matrix
\makeatletter
\renewcommand*\env@matrix[1][*\c@MaxMatrixCols c]{%
\hskip -\arraycolsep
\let\@ifnextchar\new@ifnextchar
\array{#1}}
\makeatother

% ceiling / floor
\DeclarePairedDelimiter{\ceil}{\lceil}{\rceil}
\DeclarePairedDelimiter{\floor}{\lfloor}{\rfloor}

% custom commands
\newcommand{\indefint}[2]{\int #1 \, \mathrm{d}#2}
\newcommand{\defint}[4]{\int_{#1}^{#2} #3 \, \mathrm{d}#4}
\newcommand{\pdif}[2]{\frac{\partial #1}{\partial #2}}
\newcommand{\dif}[2]{\frac{\mathrm{d}#1}{\mathrm{d}#2}}
\newcommand{\limit}[2]{\raisebox{0.5ex}{\scalebox{0.8}{$\displaystyle{\lim_{#1 \to #2}}$}}}
\newcommand{\limitsup}[2]{\raisebox{0.5ex}{\scalebox{0.8}{$\displaystyle{\limsup_{#1 \to #2}}$}}}
\newcommand{\summation}[2]{\sum\limits_{#1}^{#2}}
\newcommand{\product}[2]{\prod\limits_{#1}^{#2}}
\newcommand{\intbracket}[3]{\left[#3\right]_{#1}^{#2}}
\newcommand{\laplace}{\mathcal{L}}
\newcommand{\fourier}{\mathcal{F}}
\newcommand{\mat}[1]{\boldsymbol{#1}}
\renewcommand{\vec}[1]{\boldsymbol{#1}}
\newcommand{\rowt}[1]{\begin{bmatrix}
    #1
\end{bmatrix}^\top}
\DeclareMathOperator*{\argmax}{argmax}
\DeclareMathOperator*{\argmin}{argmin}

\newcommand{\lto}[0]{\leadsto\ }

\newcommand{\ulsmash}[1]{\underline{\smash{#1}}}

\newcommand{\powerset}[0]{\wp}
\renewcommand{\emptyset}[0]{\varnothing}

\makeatletter
\newsavebox{\@brx}
\newcommand{\llangle}[1][]{\savebox{\@brx}{\(\m@th{#1\langle}\)}%
  \mathopen{\copy\@brx\kern-0.5\wd\@brx\usebox{\@brx}}}
\newcommand{\rrangle}[1][]{\savebox{\@brx}{\(\m@th{#1\rangle}\)}%
  \mathclose{\copy\@brx\kern-0.5\wd\@brx\usebox{\@brx}}}
\makeatother
\newcommand{\lla}{\llangle}
\newcommand{\rra}{\rrangle}
\newcommand{\la}{\langle}
\newcommand{\ra}{\rangle}
\newcommand{\crnr}[1]{\text{\textopencorner} #1 \text{\textcorner}}
\newcommand{\bnfsep}[0]{\ |\ }
\newcommand{\concsep}[0]{\ ||\ }

\newcommand{\axiom}[1]{\AxiomC{#1}}
\newcommand{\unary}[1]{\UnaryInfC{#1}}
\newcommand{\binary}[1]{\BinaryInfC{#1}}
\newcommand{\trinary}[1]{\TrinaryInfC{#1}}
\newcommand{\quaternary}[1]{\QuaternaryInfC{#1}}
\newcommand{\quinary}[1]{\QuinaryInfC{#1}}
\newcommand{\dproof}[0]{\DisplayProof}
\newcommand{\llabel}[1]{\LeftLabel{\scriptsize #1}}
\newcommand{\rlabel}[1]{\RightLabel{\scriptsize #1}}

\newcommand{\ttbs}{\char`\\}
\newcommand{\lrbt}[0]{\ \bullet\ }

% colours
\newcommand{\violet}[1]{\textcolor{violet}{#1}}
\newcommand{\blue}[1]{\textcolor{blue}{#1}}
\newcommand{\red}[1]{\textcolor{red}{#1}}
\newcommand{\teal}[1]{\textcolor{teal}{#1}}

% reasoning proofs
\usepackage{ltablex}
\usepackage{environ}
\keepXColumns
\NewEnviron{reasoning}{
    \begin{tabularx}{\textwidth}{rlX}
        \BODY
    \end{tabularx}
}
\newcommand{\proofline}[3]{$(#1)$ & $#2$ & \hfill #3 \smallskip \\}
\newcommand{\proofarbitrary}[1]{& take arbitrary $#1$ \smallskip \\}
\newcommand{\prooftext}[1]{\multicolumn{3}{l}{#1} \smallskip \\}
\newcommand{\proofmath}[3]{$#1$ & = $#2$ & \hfill #3 \smallskip \\}
\newcommand{\prooftherefore}[1]{& $\therefore #1$ \smallskip \\}
\newcommand{\proofbc}[0]{\prooftext{\textbf{Base Case}}}
\newcommand{\proofis}[0]{\prooftext{\textbf{Inductive Step}}}

% ER diagrams
\newcommand{\nattribute}[4]{
    \node[draw, state, inner sep=0cm, minimum size=0.2cm, label=#3:{#4}] (#1) at (#2) {};
}
\newcommand{\mattribute}[4]{
    \node[draw, state, accepting, inner sep=0cm, minimum size=0.2cm, label=#3:{#4}] (#1) at (#2) {};
}
\newcommand{\dattribute}[4]{
    \node[draw, state, dashed, inner sep=0cm, minimum size=0.2cm, label=#3:{#4}] (#1) at (#2) {};
}
\newcommand{\entity}[3]{
    \node[] (#1-c) at (#2) {#3};
    \node[inner sep=0cm] (#1-l) at ($(#1-c) + (-1, 0)$) {};
    \node[inner sep=0cm] (#1-r) at ($(#1-c) + (1, 0)$) {};
    \node[inner sep=0cm] (#1-u) at ($(#1-c) + (0, 0.5)$) {};
    \node[inner sep=0cm] (#1-d) at ($(#1-c) + (0, -0.5)$) {};
    \draw
    ($(#1-c) + (-1, 0.5)$) -- ($(#1-c) + (1, 0.5)$) -- ($(#1-c) + (1, -0.5)$) -- ($(#1-c) + (-1, -0.5)$) -- cycle;
}
\newcommand{\relationship}[3]{
    \node[] (#1-c) at (#2) {#3};
    \node[inner sep=0cm] (#1-l) at ($(#1-c) + (-1, 0)$) {};
    \node[inner sep=0cm] (#1-r) at ($(#1-c) + (1, 0)$) {};
    \node[inner sep=0cm] (#1-u) at ($(#1-c) + (0, 1)$) {};
    \node[inner sep=0cm] (#1-d) at ($(#1-c) + (0, -1)$) {};
    \draw
    ($(#1-c) + (-1, 0)$) -- ($(#1-c) + (0, 1)$) -- ($(#1-c) + (1, 0)$) -- ($(#1-c) + (0, -1)$) -- cycle;
}

% AVL Trees
\newcommand{\avltri}[4]{
    \draw ($(#1)$) -- ($(#1) + #4*(0.5, -1)$) -- ($(#1) + #4*(-0.5, -1)$) -- cycle;
    \node at ($(#1) + #4*(0, -1) + (0, 0.5)$) {#3};
    \node at ($(#1) + #4*(0, -1) + (0, -0.5)$) {#2};
}

% RB Trees
\tikzset{rbtr/.style={inner sep=2pt, circle, draw=black, fill=red}}
\tikzset{rbtb/.style={inner sep=2pt, circle, draw=black, fill=black}}

% actual document
\begin{document}
    \section*{CO202 - Algorithms II \hfill Tutorial Sheets}
        \subsection*{Tutorial 1}
            Not sure if I'll actually cover it, since I've done these questions already in the notes.
        \subsection*{Tutorial 2}
            Note that the ~List~ class is as follows;
            \begin{lstlisting}
                class List list where
                  fromList :: [a] -> list a
                  toList :: list a -> [a]
                  normalize :: list a -> list a

                  empty :: list a
                  single :: a -> list a

                  cons :: a -> list a -> list a
                  snoc :: a -> list a -> list a
                  head :: list a -> a
                  tail :: list a -> list a
                  init :: list a -> list a
                  last :: list a -> a

                  isEmpty :: list a -> Bool
                  isSingle :: list a -> Bool

                  length :: list a -> Int
                  (++) :: list a -> list a -> list a
            \end{lstlisting}
            \begin{enumerate}[1.]
                \itemsep0em
                \item
                    The ~List~ typeclass overloads the functions ~empty~, ~cons~, ~snoc~, ~head~, ~tail~, ~init~, ~last~, ~null~, ~length~, and ~(++)~ into the ~List~ class given above.
                    It is possible to give default implementations for all of these functions.
                    For instance, the definition of ~normalize~ is
                    \begin{center}
                        $~normalize = fromList . toList~$
                    \end{center}
                    Give all the other default implementations by appropriate conversion using ~toList~ and ~fromList~;
                    \begin{lstlisting}
                        empty = fromList []
                        single x = fromList [x]

                        cons x xs = fromList (x:toList xs)
                        snoc x xs = fromList ((toList xs) ++ [x])
                        head xs = Prelude.head (toList xs)
                        tail xs = fromList (Prelude.tail (toList xs))
                        init xs = fromList (Prelude.init (toList xs))
                        last xs = Prelude.last (toList xs)

                        isEmpty xs = null (toList xs)
                        isSingle xs = case (toList xs) of [_] -> True
                                                          _   -> False

                        length xs = Prelude.length (toList xs)
                        (++) xs ys = fromList (toList xs ++ toList ys)
                    \end{lstlisting}
                \item
                    Give the trivial instance of ~List~ class for ordinary lists by giving the minimal definition of ~instance List []~.
                    \begin{lstlisting}
                        instance List [] where
                          fromList = id
                          toList = id
                          normalize = id

                          empty = []
                          single x = [x]

                          cons x xs = x:xs
                          snoc x xs = xs ++ [x]
                          head = Prelude.head
                          tail = Prelude.tail
                          init = Prelude.init
                          last = Prelude.last

                          isEmpty = null

                          isSingle [_] = True
                          isSingle _   = False

                          length = Prelude.length
                          (++) = Prelude.(++)
                    \end{lstlisting}
                \item
                    Implement the instance of the ~List~ class for the ~DList~ datatype.
                    State the complexity of each of these functions.
                    \begin{lstlisting}
                        instance List DList where
                          fromList xs = DList (xs ++)
                          toList (DList fxs) = fxs []

                          DList fxs ++ DList fys = DList (fxs . fys)
                    \end{lstlisting}
                    Generally, the time complexities are the same, except for ~tail~ (since the whole list must now be rebuilt).
                    The benefit is that ~(++)~ is now constant time.
                \item Prove that the definition of ~(++)~ for ~DList~s is correct by showing;
                    \begin{center}
                        ~fromList xs ++ fromList ys = fromList (xs ++ ys)~
                    \end{center}
                    ~fromList xs~ gives ~DList (xs ++)~, and similarly ~fromList ys~ gives ~DList (ys ++)~.
                    By our definition of ~(++)~, we know that ~fromList xs ++ fromList ys~ gives ~DList ((xs ++) . (ys ++))~.
                    Intuitively, that is equivalent to ~DList ((xs ++ ys) ++)~, which is the result of ~fromList (xs ++ ys)~.
                \item Explain the time complexity of the following definition of ~reverse~;
                    \begin{lstlisting}
                        reverse :: [a] -> [a]
                        reverse []     = []
                        reverse (x:xs) = reverse xs ++ [x]
                    \end{lstlisting}
                    This has a complexity of $O(n^2)$, due to the left nested chain of appends.
                    \begin{align*}
                        \text{let } n & = ~length xs~ & \text{for ~reverse xs~} \\
                        T_~reverse~(0) & = 1 \\
                        T_~reverse~(n) & = T_~reverse~(n - 1) + \underbrace{(n - 1)}_{T_~(++)~(n - 1)}
                    \end{align*}
                \item Show how to modify the previous definition of ~reverse~ to produce a version ~reverse' :: DList a -> DList a~, and give the time complexity of the resulting function.
                    \begin{lstlisting}
                        reverse' :: DList a -> DList a
                        reverse' xs
                          | isEmpty xs = empty
                          | otherwise  = reverse' (tail xs) ++ single (head xs)
                    \end{lstlisting}
                    This has a time complexity in $O(n)$, as ~(++)~ is right associated.
                \item Give a trivial representation of lists where ~length~ takes $O(1)$, and that does not affect the complexity of other operations.
                    \begin{lstlisting}
                        data LList a = LList Int [a]

                        instance List LList where
                          fromList xs = LList (length xs) xs
                          toList (LList _ xs) = xs
                          cons x (LList n xs) = LList (n + 1) (x:xs)
                          length (LList n _) = n
                    \end{lstlisting}
                    This simply stores the length of the list as a parameter.
            \end{enumerate}
\end{document}