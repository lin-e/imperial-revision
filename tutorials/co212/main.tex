\documentclass[a4paper, 12pt]{article}

% packages
\usepackage{amssymb}
\usepackage[fleqn]{mathtools}
\usepackage{tikz}
\usepackage{enumerate}
\usepackage{bussproofs}
\usepackage{xcolor}
\usepackage[margin=1.3cm]{geometry}
\usepackage{logicproof}
\usepackage{diagbox}
\usepackage{listings}
\usepackage{graphicx}
\usepackage{lstautogobble}
\usepackage{hyperref}
\usepackage{multirow}
\usepackage{tipa}
\usepackage{pgfplots}

% tikz libraries
\usetikzlibrary{
    decorations.pathreplacing,
    arrows,
    shapes.gates.logic.US,
    circuits.logic.US,
    calc,
    automata,
    positioning,
    intersections
}

\pgfplotsset{compat=1.16}

\pgfmathdeclarefunction{gauss}{2}{%
  \pgfmathparse{1/(#2*sqrt(2*pi))*exp(-((x-#1)^2)/(2*#2^2))}%
}

\allowdisplaybreaks % allow environments to break
\setlength\parindent{0pt} % no indent

% shorthand for verbatim
% this clashes with logicproof, so maybe fix this at some point?
\catcode`~=\active
\def~#1~{\texttt{#1}}

% code listing
\lstdefinestyle{main}{
    numberstyle=\tiny,
    breaklines=true,
    showspaces=false,
    showstringspaces=false,
    tabsize=2,
    numbers=left,
    basicstyle=\ttfamily,
    columns=fixed,
    fontadjust=true,
    basewidth=0.5em,
    autogobble,
    xleftmargin=3.0ex,
    mathescape=true
}
\newcommand{\dollar}{\mbox{\textdollar}} %
\lstset{style=main}

% augmented matrix
\makeatletter
\renewcommand*\env@matrix[1][*\c@MaxMatrixCols c]{%
\hskip -\arraycolsep
\let\@ifnextchar\new@ifnextchar
\array{#1}}
\makeatother

% ceiling / floor
\DeclarePairedDelimiter{\ceil}{\lceil}{\rceil}
\DeclarePairedDelimiter{\floor}{\lfloor}{\rfloor}

% custom commands
\newcommand{\indefint}[2]{\int #1 \, \mathrm{d}#2}
\newcommand{\defint}[4]{\int_{#1}^{#2} #3 \, \mathrm{d}#4}
\newcommand{\pdif}[2]{\frac{\partial #1}{\partial #2}}
\newcommand{\dif}[2]{\frac{\mathrm{d}#1}{\mathrm{d}#2}}
\newcommand{\limit}[2]{\raisebox{0.5ex}{\scalebox{0.8}{$\displaystyle{\lim_{#1 \to #2}}$}}}
\newcommand{\limitsup}[2]{\raisebox{0.5ex}{\scalebox{0.8}{$\displaystyle{\limsup_{#1 \to #2}}$}}}
\newcommand{\summation}[2]{\sum\limits_{#1}^{#2}}
\newcommand{\product}[2]{\prod\limits_{#1}^{#2}}
\newcommand{\intbracket}[3]{\left[#3\right]_{#1}^{#2}}
\newcommand{\laplace}{\mathcal{L}}
\newcommand{\fourier}{\mathcal{F}}
\newcommand{\mat}[1]{\boldsymbol{#1}}
\renewcommand{\vec}[1]{\boldsymbol{#1}}
\newcommand{\rowt}[1]{\begin{bmatrix}
    #1
\end{bmatrix}^\top}
\DeclareMathOperator*{\argmax}{argmax}
\DeclareMathOperator*{\argmin}{argmin}

\newcommand{\lto}[0]{\leadsto\ }

\newcommand{\ulsmash}[1]{\underline{\smash{#1}}}

\newcommand{\powerset}[0]{\wp}
\renewcommand{\emptyset}[0]{\varnothing}

\makeatletter
\newsavebox{\@brx}
\newcommand{\llangle}[1][]{\savebox{\@brx}{\(\m@th{#1\langle}\)}%
  \mathopen{\copy\@brx\kern-0.5\wd\@brx\usebox{\@brx}}}
\newcommand{\rrangle}[1][]{\savebox{\@brx}{\(\m@th{#1\rangle}\)}%
  \mathclose{\copy\@brx\kern-0.5\wd\@brx\usebox{\@brx}}}
\makeatother
\newcommand{\lla}{\llangle}
\newcommand{\rra}{\rrangle}
\newcommand{\la}{\langle}
\newcommand{\ra}{\rangle}
\newcommand{\crnr}[1]{\text{\textopencorner} #1 \text{\textcorner}}
\newcommand{\bnfsep}[0]{\ |\ }
\newcommand{\concsep}[0]{\ ||\ }

\newcommand{\axiom}[1]{\AxiomC{#1}}
\newcommand{\unary}[1]{\UnaryInfC{#1}}
\newcommand{\binary}[1]{\BinaryInfC{#1}}
\newcommand{\trinary}[1]{\TrinaryInfC{#1}}
\newcommand{\quaternary}[1]{\QuaternaryInfC{#1}}
\newcommand{\quinary}[1]{\QuinaryInfC{#1}}
\newcommand{\dproof}[0]{\DisplayProof}

\newcommand{\ttbs}{\char`\\}
\newcommand{\lrbt}[0]{\ \bullet\ }

% colours
\newcommand{\violet}[1]{\textcolor{violet}{#1}}
\newcommand{\blue}[1]{\textcolor{blue}{#1}}
\newcommand{\red}[1]{\textcolor{red}{#1}}
\newcommand{\teal}[1]{\textcolor{teal}{#1}}

% reasoning proofs
\usepackage{ltablex}
\usepackage{environ}
\keepXColumns
\NewEnviron{reasoning}{
    \begin{tabularx}{\textwidth}{rlX}
        \BODY
    \end{tabularx}
}
\newcommand{\proofline}[3]{$(#1)$ & $#2$ & \hfill #3 \smallskip \\}
\newcommand{\proofarbitrary}[1]{& take arbitrary $#1$ \smallskip \\}
\newcommand{\prooftext}[1]{\multicolumn{3}{l}{#1} \smallskip \\}
\newcommand{\proofmath}[3]{$#1$ & = $#2$ & \hfill #3 \smallskip \\}
\newcommand{\prooftherefore}[1]{& $\therefore #1$ \smallskip \\}
\newcommand{\proofbc}[0]{\prooftext{\textbf{Base Case}}}
\newcommand{\proofis}[0]{\prooftext{\textbf{Inductive Step}}}

% ER diagrams
\newcommand{\nattribute}[4]{
    \node[draw, state, inner sep=0cm, minimum size=0.2cm, label=#3:{#4}] (#1) at (#2) {};
}
\newcommand{\mattribute}[4]{
    \node[draw, state, accepting, inner sep=0cm, minimum size=0.2cm, label=#3:{#4}] (#1) at (#2) {};
}
\newcommand{\dattribute}[4]{
    \node[draw, state, dashed, inner sep=0cm, minimum size=0.2cm, label=#3:{#4}] (#1) at (#2) {};
}
\newcommand{\entity}[3]{
    \node[] (#1-c) at (#2) {#3};
    \node[inner sep=0cm] (#1-l) at ($(#1-c) + (-1, 0)$) {};
    \node[inner sep=0cm] (#1-r) at ($(#1-c) + (1, 0)$) {};
    \node[inner sep=0cm] (#1-u) at ($(#1-c) + (0, 0.5)$) {};
    \node[inner sep=0cm] (#1-d) at ($(#1-c) + (0, -0.5)$) {};
    \draw
    ($(#1-c) + (-1, 0.5)$) -- ($(#1-c) + (1, 0.5)$) -- ($(#1-c) + (1, -0.5)$) -- ($(#1-c) + (-1, -0.5)$) -- cycle;
}
\newcommand{\relationship}[3]{
    \node[] (#1-c) at (#2) {#3};
    \node[inner sep=0cm] (#1-l) at ($(#1-c) + (-1, 0)$) {};
    \node[inner sep=0cm] (#1-r) at ($(#1-c) + (1, 0)$) {};
    \node[inner sep=0cm] (#1-u) at ($(#1-c) + (0, 1)$) {};
    \node[inner sep=0cm] (#1-d) at ($(#1-c) + (0, -1)$) {};
    \draw
    ($(#1-c) + (-1, 0)$) -- ($(#1-c) + (0, 1)$) -- ($(#1-c) + (1, 0)$) -- ($(#1-c) + (0, -1)$) -- cycle;
}

% AVL Trees
\newcommand{\avltri}[4]{
    \draw ($(#1)$) -- ($(#1) + #4*(0.5, -1)$) -- ($(#1) + #4*(-0.5, -1)$) -- cycle;
    \node at ($(#1) + #4*(0, -1) + (0, 0.5)$) {#3};
    \node at ($(#1) + #4*(0, -1) + (0, -0.5)$) {#2};
}

% RB Trees
\tikzset{rbtr/.style={inner sep=2pt, circle, draw=black, fill=red}}
\tikzset{rbtb/.style={inner sep=2pt, circle, draw=black, fill=black}}

% actual document
\begin{document}
    \section*{CO212 - Networks and Communications \hfill Tutorial Sheets}
        \subsection*{Tutorial 1 - Basic Concepts}
            \begin{enumerate}[1.]
                \itemsep0em
                \item
                    Consider transferring a 1 GB tape using the following mediums.
                    Which is faster, i.e. has a higher data rate?
                    \begin{enumerate}[(a)]
                        \itemsep0em
                        \item A 56 Kbps modem
                        \item Next-day delivery through the postal system
                    \end{enumerate}
                    The modem has a transfer time of
                    $$\frac{L}{R} = \frac{1 \times 10^9 \times 8}{56 \times 10^3} \approx 142857 \text{ seconds} \approx 39.68 \text{ hours}$$
                    Compared to the postal system, which takes 24 hours, the postal system is clearly faster.
                    However, the postal system has a 24 hour latency (the first bit takes 24 hours to arrive), whereas the modem has very low latency (relatively).
                \item Would you use a connectionless or connection-oriented network
                    \begin{enumerate}[(a)]
                        \itemsep0em
                        \item if the underlying network suffers from frequent congested paths? \hfill connectionless
                            \medskip

                            Provides flexibility for routing around congestion.
                        \item for a video conferencing application? \hfill connection-oriented
                            \medskip

                            We want to reserve guaranteed resources, as we want low-latency.
                            The overhead is justified as it will be used for a long-term connection.
                        \item for a short message transfer? \hfill connectionless
                            \medskip

                            We want to avoid the setup overhead found in connection-oriented networks.
                    \end{enumerate}
                \item
                    Consider two hosts, $A$ and $B$, connected by a single link of rate $R$ bps.
                    Suppose that the two hosts are separated by $m$ metres and suppose that the propagation speed along the link is $s$ metres/sec.
                    Host $A$ is to send a packet of size $L$ bits to host $B$.
                    \begin{enumerate}[(a)]
                        \itemsep0em
                        \item Express the propagation delay $d_\text{prop}$ in terms of $m$ and $s$.
                            $$\frac{m}{s}$$
                        \item Determine the transmission time of the packet $d_\text{tran}$ in terms of $L$ and $R$.
                            $$\frac{L}{R}$$
                        \item Ignoring processing and queueing delay, obtain an expression for the end-to-end delay $d_\text{end-to-end}$.
                            $$\frac{m}{s} + \frac{L}{R}$$
                        \item
                            Suppose host $A$ begins to transmit the packet at time $t = 0$.
                            At time $t = d_\text{tran}$, where is the last bit of the packet?
                            \medskip

                            Leaving host $A$.
                        \item
                            Suppose $d_\text{prop}$ is greater than $d_\text{tran}$.
                            At time $t = d_\text{tran}$, where is the first bit of the packet?
                            \medskip

                            In the link, has not reached host $B$.
                        \item
                            Suppose $d_\text{prop}$ is smaller than $d_\text{tran}$.
                            At time $t = d_\text{tran}$, where is the first bit of the packet?
                            \medskip

                            At host $B$.
                        \item
                            Suppose $s = 2.5 \times 10^8$, $L = 120$ bits, and $R = 56$ Kbps.
                            Find the distance $m$ so that $d_\text{prop}$ equals $d_\text{tran}$.
                            $$\frac{m}{s} = \frac{L}{R} \Rightarrow \frac{m}{2.5 \times 10^8} = \frac{120}{56 \times 10^3} \Rightarrow m = \frac{120 \cdot 2.5 \times 10^8}{56 \times 10^3} \approx 535714.3 \text{ m}$$
                    \end{enumerate}
                \item
                    Suppose two hosts, $A$ and $B$, are separated by $20,000$ Km, and are connected by a direct link of $R = 2$ Mbps.
                    Suppose that the propagation speed over the link is $2.5 \times 10^8$ metres/sec.
                    \begin{enumerate}[(a)]
                        \itemsep0em
                        \item Calculate the bandwidth-delay product, $R \cdot d_\text{prop}$.
                            $$R \cdot d_\text{prop} = 2 \times 10^6 \cdot \frac{20000 \times 10^3}{2.5 \times 10^8} = 160000 \text{ bits}$$
                        \item
                            Consider as ending a file of $800,000$ bits from $A$ to $B$.
                            Suppose the file is sent continuously as one large message.
                            What is the maximum number of bits that will be in the link at any given time?
                            \medskip

                            16000 bits
                        \item Provide an interpretation of the bandwidth-delay product.
                            \medskip

                            The number of bits that can be on the link at any time.
                        \item
                            What is the width (in metres) of a bit in the link?
                            Is it longer than a football field ($\approx 105$ metres)?
                            \medskip

                            Given the link is 20000 Km, and it can fit 16000 bits, each bit is 125 metres, hence it is longer than a football field.
                        \item Derive a general expression for the width of a bit in terms of the propagation speed $s$, the transmission rate $R$, and the length of the link $m$.
                            $$\frac{m}{R \cdot d_\text{prop}} = \frac{m}{R \cdot \frac{m}{s}} = \frac{s}{R}$$
                        \item
                            Suppose we can modify $R$.
                            For what value of $R$ is the width of a bit as long as the length of the link?
                            \medskip

                            Using the expression above, we can solve for $R$;
                            $$\frac{s}{R} = m \Rightarrow R = \frac{s}{m}$$
                    \end{enumerate}
            \end{enumerate}
\end{document}