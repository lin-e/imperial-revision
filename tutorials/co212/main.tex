\documentclass[a4paper, 12pt]{article}

% packages
\usepackage{amssymb}
\usepackage[fleqn]{mathtools}
\usepackage{tikz}
\usepackage{enumerate}
\usepackage{bussproofs}
\usepackage{xcolor}
\usepackage[margin=1.3cm]{geometry}
\usepackage{logicproof}
\usepackage{diagbox}
\usepackage{listings}
\usepackage{graphicx}
\usepackage{lstautogobble}
\usepackage{hyperref}
\usepackage{multirow}
\usepackage{tipa}
\usepackage{pgfplots}

% tikz libraries
\usetikzlibrary{
    decorations.pathreplacing,
    arrows,
    shapes.gates.logic.US,
    circuits.logic.US,
    calc,
    automata,
    positioning,
    intersections
}

\pgfplotsset{compat=1.16}

\pgfmathdeclarefunction{gauss}{2}{%
  \pgfmathparse{1/(#2*sqrt(2*pi))*exp(-((x-#1)^2)/(2*#2^2))}%
}

\allowdisplaybreaks % allow environments to break
\setlength\parindent{0pt} % no indent

% shorthand for verbatim
% this clashes with logicproof, so maybe fix this at some point?
\catcode`~=\active
\def~#1~{\texttt{#1}}

% code listing
\lstdefinestyle{main}{
    numberstyle=\tiny,
    breaklines=true,
    showspaces=false,
    showstringspaces=false,
    tabsize=2,
    numbers=left,
    basicstyle=\ttfamily,
    columns=fixed,
    fontadjust=true,
    basewidth=0.5em,
    autogobble,
    xleftmargin=3.0ex,
    mathescape=true
}
\newcommand{\dollar}{\mbox{\textdollar}} %
\lstset{style=main}

% augmented matrix
\makeatletter
\renewcommand*\env@matrix[1][*\c@MaxMatrixCols c]{%
\hskip -\arraycolsep
\let\@ifnextchar\new@ifnextchar
\array{#1}}
\makeatother

% ceiling / floor
\DeclarePairedDelimiter{\ceil}{\lceil}{\rceil}
\DeclarePairedDelimiter{\floor}{\lfloor}{\rfloor}

% custom commands
\newcommand{\indefint}[2]{\int #1 \, \mathrm{d}#2}
\newcommand{\defint}[4]{\int_{#1}^{#2} #3 \, \mathrm{d}#4}
\newcommand{\pdif}[2]{\frac{\partial #1}{\partial #2}}
\newcommand{\dif}[2]{\frac{\mathrm{d}#1}{\mathrm{d}#2}}
\newcommand{\limit}[2]{\raisebox{0.5ex}{\scalebox{0.8}{$\displaystyle{\lim_{#1 \to #2}}$}}}
\newcommand{\limitsup}[2]{\raisebox{0.5ex}{\scalebox{0.8}{$\displaystyle{\limsup_{#1 \to #2}}$}}}
\newcommand{\summation}[2]{\sum\limits_{#1}^{#2}}
\newcommand{\product}[2]{\prod\limits_{#1}^{#2}}
\newcommand{\intbracket}[3]{\left[#3\right]_{#1}^{#2}}
\newcommand{\laplace}{\mathcal{L}}
\newcommand{\fourier}{\mathcal{F}}
\newcommand{\mat}[1]{\boldsymbol{#1}}
\renewcommand{\vec}[1]{\boldsymbol{#1}}
\newcommand{\rowt}[1]{\begin{bmatrix}
    #1
\end{bmatrix}^\top}
\DeclareMathOperator*{\argmax}{argmax}
\DeclareMathOperator*{\argmin}{argmin}

\newcommand{\lto}[0]{\leadsto\ }

\newcommand{\ulsmash}[1]{\underline{\smash{#1}}}

\newcommand{\powerset}[0]{\wp}
\renewcommand{\emptyset}[0]{\varnothing}

\makeatletter
\newsavebox{\@brx}
\newcommand{\llangle}[1][]{\savebox{\@brx}{\(\m@th{#1\langle}\)}%
  \mathopen{\copy\@brx\kern-0.5\wd\@brx\usebox{\@brx}}}
\newcommand{\rrangle}[1][]{\savebox{\@brx}{\(\m@th{#1\rangle}\)}%
  \mathclose{\copy\@brx\kern-0.5\wd\@brx\usebox{\@brx}}}
\makeatother
\newcommand{\lla}{\llangle}
\newcommand{\rra}{\rrangle}
\newcommand{\la}{\langle}
\newcommand{\ra}{\rangle}
\newcommand{\crnr}[1]{\text{\textopencorner} #1 \text{\textcorner}}
\newcommand{\bnfsep}[0]{\ |\ }
\newcommand{\concsep}[0]{\ ||\ }

\newcommand{\axiom}[1]{\AxiomC{#1}}
\newcommand{\unary}[1]{\UnaryInfC{#1}}
\newcommand{\binary}[1]{\BinaryInfC{#1}}
\newcommand{\trinary}[1]{\TrinaryInfC{#1}}
\newcommand{\quaternary}[1]{\QuaternaryInfC{#1}}
\newcommand{\quinary}[1]{\QuinaryInfC{#1}}
\newcommand{\dproof}[0]{\DisplayProof}

\newcommand{\ttbs}{\char`\\}
\newcommand{\lrbt}[0]{\ \bullet\ }

% colours
\newcommand{\violet}[1]{\textcolor{violet}{#1}}
\newcommand{\blue}[1]{\textcolor{blue}{#1}}
\newcommand{\red}[1]{\textcolor{red}{#1}}
\newcommand{\teal}[1]{\textcolor{teal}{#1}}

% reasoning proofs
\usepackage{ltablex}
\usepackage{environ}
\keepXColumns
\NewEnviron{reasoning}{
    \begin{tabularx}{\textwidth}{rlX}
        \BODY
    \end{tabularx}
}
\newcommand{\proofline}[3]{$(#1)$ & $#2$ & \hfill #3 \smallskip \\}
\newcommand{\proofarbitrary}[1]{& take arbitrary $#1$ \smallskip \\}
\newcommand{\prooftext}[1]{\multicolumn{3}{l}{#1} \smallskip \\}
\newcommand{\proofmath}[3]{$#1$ & = $#2$ & \hfill #3 \smallskip \\}
\newcommand{\prooftherefore}[1]{& $\therefore #1$ \smallskip \\}
\newcommand{\proofbc}[0]{\prooftext{\textbf{Base Case}}}
\newcommand{\proofis}[0]{\prooftext{\textbf{Inductive Step}}}

% ER diagrams
\newcommand{\nattribute}[4]{
    \node[draw, state, inner sep=0cm, minimum size=0.2cm, label=#3:{#4}] (#1) at (#2) {};
}
\newcommand{\mattribute}[4]{
    \node[draw, state, accepting, inner sep=0cm, minimum size=0.2cm, label=#3:{#4}] (#1) at (#2) {};
}
\newcommand{\dattribute}[4]{
    \node[draw, state, dashed, inner sep=0cm, minimum size=0.2cm, label=#3:{#4}] (#1) at (#2) {};
}
\newcommand{\entity}[3]{
    \node[] (#1-c) at (#2) {#3};
    \node[inner sep=0cm] (#1-l) at ($(#1-c) + (-1, 0)$) {};
    \node[inner sep=0cm] (#1-r) at ($(#1-c) + (1, 0)$) {};
    \node[inner sep=0cm] (#1-u) at ($(#1-c) + (0, 0.5)$) {};
    \node[inner sep=0cm] (#1-d) at ($(#1-c) + (0, -0.5)$) {};
    \draw
    ($(#1-c) + (-1, 0.5)$) -- ($(#1-c) + (1, 0.5)$) -- ($(#1-c) + (1, -0.5)$) -- ($(#1-c) + (-1, -0.5)$) -- cycle;
}
\newcommand{\relationship}[3]{
    \node[] (#1-c) at (#2) {#3};
    \node[inner sep=0cm] (#1-l) at ($(#1-c) + (-1, 0)$) {};
    \node[inner sep=0cm] (#1-r) at ($(#1-c) + (1, 0)$) {};
    \node[inner sep=0cm] (#1-u) at ($(#1-c) + (0, 1)$) {};
    \node[inner sep=0cm] (#1-d) at ($(#1-c) + (0, -1)$) {};
    \draw
    ($(#1-c) + (-1, 0)$) -- ($(#1-c) + (0, 1)$) -- ($(#1-c) + (1, 0)$) -- ($(#1-c) + (0, -1)$) -- cycle;
}

% AVL Trees
\newcommand{\avltri}[4]{
    \draw ($(#1)$) -- ($(#1) + #4*(0.5, -1)$) -- ($(#1) + #4*(-0.5, -1)$) -- cycle;
    \node at ($(#1) + #4*(0, -1) + (0, 0.5)$) {#3};
    \node at ($(#1) + #4*(0, -1) + (0, -0.5)$) {#2};
}

% RB Trees
\tikzset{rbtr/.style={inner sep=2pt, circle, draw=black, fill=red}}
\tikzset{rbtb/.style={inner sep=2pt, circle, draw=black, fill=black}}

% actual document
\begin{document}
    \section*{CO212 - Networks and Communications \hfill Tutorial Sheets}
        \subsection*{Tutorial 1 - Basic Concepts}
            \begin{enumerate}[1.]
                \itemsep0em
                \item
                    Consider transferring a 1 GB tape using the following mediums.
                    Which is faster, i.e. has a higher data rate?
                    \begin{enumerate}[(a)]
                        \itemsep0em
                        \item A 56 Kbps modem
                        \item Next-day delivery through the postal system
                    \end{enumerate}
                    The modem has a transfer time of
                    $$\frac{L}{R} = \frac{1 \times 10^9 \times 8}{56 \times 10^3} \approx 142857 \text{ seconds} \approx 39.68 \text{ hours}$$
                    Compared to the postal system, which takes 24 hours, the postal system is clearly faster.
                    However, the postal system has a 24 hour latency (the first bit takes 24 hours to arrive), whereas the modem has very low latency (relatively).
                \item Would you use a connectionless or connection-oriented network
                    \begin{enumerate}[(a)]
                        \itemsep0em
                        \item if the underlying network suffers from frequent congested paths? \hfill connectionless
                            \medskip

                            Provides flexibility for routing around congestion.
                        \item for a video conferencing application? \hfill connection-oriented
                            \medskip

                            We want to reserve guaranteed resources, as we want low-latency.
                            The overhead is justified as it will be used for a long-term connection.
                        \item for a short message transfer? \hfill connectionless
                            \medskip

                            We want to avoid the setup overhead found in connection-oriented networks.
                    \end{enumerate}
                \item
                    Consider two hosts, $A$ and $B$, connected by a single link of rate $R$ bps.
                    Suppose that the two hosts are separated by $m$ metres and suppose that the propagation speed along the link is $s$ metres/sec.
                    Host $A$ is to send a packet of size $L$ bits to host $B$.
                    \begin{enumerate}[(a)]
                        \itemsep0em
                        \item Express the propagation delay $d_\text{prop}$ in terms of $m$ and $s$.
                            $$\frac{m}{s}$$
                        \item Determine the transmission time of the packet $d_\text{tran}$ in terms of $L$ and $R$.
                            $$\frac{L}{R}$$
                        \item Ignoring processing and queueing delay, obtain an expression for the end-to-end delay $d_\text{end-to-end}$.
                            $$\frac{m}{s} + \frac{L}{R}$$
                        \item
                            Suppose host $A$ begins to transmit the packet at time $t = 0$.
                            At time $t = d_\text{tran}$, where is the last bit of the packet?
                            \medskip

                            Leaving host $A$.
                        \item
                            Suppose $d_\text{prop}$ is greater than $d_\text{tran}$.
                            At time $t = d_\text{tran}$, where is the first bit of the packet?
                            \medskip

                            In the link, has not reached host $B$.
                        \item
                            Suppose $d_\text{prop}$ is smaller than $d_\text{tran}$.
                            At time $t = d_\text{tran}$, where is the first bit of the packet?
                            \medskip

                            At host $B$.
                        \item
                            Suppose $s = 2.5 \times 10^8$, $L = 120$ bits, and $R = 56$ Kbps.
                            Find the distance $m$ so that $d_\text{prop}$ equals $d_\text{tran}$.
                            $$\frac{m}{s} = \frac{L}{R} \Rightarrow \frac{m}{2.5 \times 10^8} = \frac{120}{56 \times 10^3} \Rightarrow m = \frac{120 \cdot 2.5 \times 10^8}{56 \times 10^3} \approx 535714.3 \text{ m}$$
                    \end{enumerate}
                \item
                    Suppose two hosts, $A$ and $B$, are separated by $20,000$ Km, and are connected by a direct link of $R = 2$ Mbps.
                    Suppose that the propagation speed over the link is $2.5 \times 10^8$ metres/sec.
                    \begin{enumerate}[(a)]
                        \itemsep0em
                        \item Calculate the bandwidth-delay product, $R \cdot d_\text{prop}$.
                            $$R \cdot d_\text{prop} = 2 \times 10^6 \cdot \frac{20000 \times 10^3}{2.5 \times 10^8} = 160000 \text{ bits}$$
                        \item
                            Consider as ending a file of $800,000$ bits from $A$ to $B$.
                            Suppose the file is sent continuously as one large message.
                            What is the maximum number of bits that will be in the link at any given time?
                            \medskip

                            16000 bits
                        \item Provide an interpretation of the bandwidth-delay product.
                            \medskip

                            The number of bits that can be on the link at any time.
                        \item
                            What is the width (in metres) of a bit in the link?
                            Is it longer than a football field ($\approx 105$ metres)?
                            \medskip

                            Given the link is 20000 Km, and it can fit 16000 bits, each bit is 125 metres, hence it is longer than a football field.
                        \item Derive a general expression for the width of a bit in terms of the propagation speed $s$, the transmission rate $R$, and the length of the link $m$.
                            $$\frac{m}{R \cdot d_\text{prop}} = \frac{m}{R \cdot \frac{m}{s}} = \frac{s}{R}$$
                        \item
                            Suppose we can modify $R$.
                            For what value of $R$ is the width of a bit as long as the length of the link?
                            \medskip

                            Using the expression above, we can solve for $R$;
                            $$\frac{s}{R} = m \Rightarrow R = \frac{s}{m}$$
                    \end{enumerate}
            \end{enumerate}
        \subsection*{Tutorial 2 - Application Layer}
            \begin{enumerate}[1.]
                \itemsep0em
                \item Consider the following scenario when from within your Web browser you click on a link to obtain a webpage.
                    \begin{itemize}
                        \itemsep0em
                        \item
                            The IP address for the associated URL is not cached in your local host, so a DNS lookup is necessary to obtain the IP address.
                            Suppose that $n$ DNS servers are visited before your host receives the IP address from the DNS; visiting $k$ of them incurs a RTT of $D_1$ per DNS, and visiting each of the remaining incurs an RTT of $D_2$.
                        \item The webpage associated with the link contains $m$ small objects.
                        \item HTTP is running in non-persistent mode.
                        \item $\text{RTT}_0$ denotes the RTT between the local host and the server for each object.
                    \end{itemize}
                    Assuming zero transmission time of each object, calculate the amount of time that elapses from when the client clicks on the link until the client receives all the objects.
                    \begin{center}
                        $(m + 1) \cdot 2 \cdot \text{RTT}_0 + k \cdot D_1 + (n - k) \cdot D_2$
                    \end{center}
                    Note that $\text{RTT}_0$ is multiplied by 2, as we need the time to open a connection, and \textbf{then} the time to download each object.
                    $(m + 1)$ is the $m$ objects, as well as the main page.
                \item
                    Referring to the previous question, suppose three DNS servers are visited and the value of $k$ is 2.
                    Further, the HTML file references five very small objects on the same server.
                    Neglecting transmission times, how much time elapses with;
                    \begin{enumerate}[(a)]
                        \itemsep0em
                        \item Non-persistent HTTP with no parallel TCP connections?
                            \begin{center}
                                $12 \cdot \text{RTT}_0 + 2 \cdot D_1 + D_2$
                            \end{center}
                            This is essentially using the expression derived above, substituting values when appropriate.
                        \item Non-persistent HTTP with the browser configured for five parallel connections?
                            \begin{center}
                                $4 \cdot \text{RTT}_0 + 2 \cdot D_1 + D_2$
                            \end{center}
                            This is similar to the above, but since we have 5 parallel connections, we can do a batch of 5 (simultaneously), and then 1 by itself (total of 6).
                        \item Persistent HTTP connection?
                            \begin{center}
                                $7 \cdot \text{RTT}_0 + 2 \cdot D_1 + D_2$
                            \end{center}
                            This has an initial $\text{RTT}_0$ to open the connection, and then another 6 for the content.
                    \end{enumerate}
                \item
                    Consider a short, 15-meter link, over which a sender can transmit at a rate of $150$ bps in both directions.
                    Suppose that packets containing data are $200,000$ bits long.
                    Assume that $N$ parallel connections each get $\frac{1}{N}$ of the link bandwidth.
                    Now consider the HTTP protocol, and suppose that each downloaded object is 200Kb long, and that the initial downloaded object contains 10 referenced objects from the same sender.
                    Would parallel downloads via parallel instances of non-persistent HTTP make sense in this case?
                    Now consider persistent HTTP.
                    Do you expect significant gains over the non-persistent case?
                    Justify and explain your answer.
                    \medskip

                    First, consider the parallel case - the initial connection has a link bandwidth of $150$ bps, and the 10 parallel connections each have a link bandwidth of $15$ bps.
                    Let $d_p$ be the time it takes to send a request.
                    \begin{align*}
                        T & = \overbrace{\underbrace{2 \cdot d_p}_\text{(1)} + \underbrace{d_p}_\text{(2)} + \underbrace{d_p + \frac{200000}{150}}_\text{(3)}}^\text{initial object} + \overbrace{\underbrace{2 \cdot d_p}_\text{(1)} + \underbrace{d_p}_\text{(2)} + \underbrace{d_p + \frac{200000}{15}}_\text{(4)}}^\text{10 parallel objects} \\
                        & = 8 \cdot d_p + \frac{2000000}{150} + \frac{2000000}{15} \\
                        & \approx 8 \cdot d_p + 14666.67 \text{ seconds}
                    \end{align*}
                    \begin{enumerate}[(1)]
                        \itemsep0em
                        \item open the connection
                        \item send request for object
                        \item download object with link bandwidth of $150$ bps
                        \item download object with link bandwidth of $15$ bps
                    \end{enumerate}
                    Now we can do the same for a persistent HTTP connection, without parallel requests;
                    \begin{align*}
                        T & = \overbrace{\underbrace{2 \cdot d_p}_\text{(1)} + \underbrace{d_p}_\text{(2)} + \underbrace{d_p + \frac{200000}{150}}_\text{(3)}}^\text{initial object} + 10 \cdot \overbrace{(\underbrace{d_p}_\text{(2)} + \underbrace{d_p + \frac{200000}{15}}_\text{(3)})}^\text{single object} \\
                        & = 24 \cdot d_p + \frac{2000000}{150} + 10 \cdot \frac{2000000}{150} \\
                        & \approx 24 \cdot d_p + 14666.67 \text{ seconds}
                    \end{align*}
                    We can calculate $d_p$ as $\frac{15}{c} = \frac{15}{3 \times 10^8} = 0.05 \mu \text{s}$, which is negligible.
                    Therefore, persistent HTTP approximately achieves the same download times as non-persistent HTTP with parallel downloads.
                \item
                    Consider the scenario introduced in the previous exercise.
                    Now suppose that the link is shared by Bob with four other users.
                    Bob uses parallel instances of non-persistent HTTP and the four other users use use non-persistent HTTP without parallel downloads.
                    \begin{enumerate}[(a)]
                        \itemsep0em
                        \item Do Bob's parallel connections help him get webpages more quickly?
                            \medskip

                            Yes, as the use of more parallel connections gives him a larger share of the link bandwidth.
                        \item If all five users open five parallel instances of non-persistent HTTP, then would Bob's parallel connections still be beneficial?
                            \medskip

                            Yes, as he would have a smaller share of the link bandwidth if he wasn't using parallel connections.
                    \end{enumerate}
                \item
                    The DNS server in your domain is updated when the mail server is (temporarily) moved to another machine during a systems upgrade.
                    Users continue to use the name mail however.
                    Lookups you make on mail will return a variety of information.
                    \begin{enumerate}[(a)]
                        \itemsep0em
                        \item What information would DNS return?
                            \medskip

                            ~CNAME~ records mapping mail to a hostname, which will change during an upgrade, and ~A~ record mapping a hostname to an IP address.
                        \item What value ranges would you expect the TTL to take before, during and after the mail server migration?
                            \medskip

                            The ~A~ record can have a long TTL, such as 86400 seconds, but the ~CNAME~ record should be low, such as as 6000 seconds.
                    \end{enumerate}
                \item What information does the URL ~http://www.phdcomics.com:80/comics.php~ give?
                    \begin{itemize}
                        \itemsep0em
                        \item ~http~ \hfill use HTTP
                        \item ~www.phdcompics.com~ \hfill gives hostname (which can be resolved to an IP)
                        \item ~:80~ \hfill connect on port 80
                        \item ~/comics.php~ \hfill name of resource is a PHP script, suggesting dynamic content
                    \end{itemize}
                \item
                    Suppose a host elects to use a name server not within its organisation for address resolution.
                    When would this result in no more total traffic, assuming queries are not found in the DNS caches, than with a local name server?
                    When might this result in a better DNS cache hit rate and possibly less total traffic?
            \end{enumerate}
\end{document}