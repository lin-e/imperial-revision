\documentclass[a4paper, 12pt]{article}
    % packages
    \usepackage{amssymb}
    \usepackage[fleqn]{mathtools}
    \usepackage{tikz}
    \usepackage{enumerate}
    \usepackage{bussproofs}
    \usepackage[margin=1.3cm]{geometry}

    % augmented matrix
    \makeatletter
    \renewcommand*\env@matrix[1][*\c@MaxMatrixCols c]{%
    \hskip -\arraycolsep
    \let\@ifnextchar\new@ifnextchar
    \array{#1}}
    \makeatother

    % ceiling / floor
    \DeclarePairedDelimiter{\ceil}{\lceil}{\rceil}
    \DeclarePairedDelimiter{\floor}{\lfloor}{\rfloor}

    % custom commands
    \newcommand{\indefint}[2]{\int #1 \, \mathrm{d}#2}
    \newcommand{\defint}[4]{\int_#1^#2 #3 \, \mathrm{d}#4}
    \newcommand{\dif}[2]{\frac{\mathrm{d}#1}{\mathrm{d}#2}}
    \newcommand{\limit}[2]{\displaystyle{\lim_{#1 \to #2}}}
    \newcommand{\summation}[3]{\sum\limits_{#1}^#2 #3}
    \newcommand{\intbracket}[3]{\left[#3\right]_#1^#2}

    \newcommand{\powerset}[0]{\wp}
    \renewcommand{\emptyset}[0]{\varnothing}

    \newcommand{\unaryproof}[2]{\AxiomC{#1} \UnaryInfC{#2} \DisplayProof}
    \newcommand{\binaryproof}[3]{\AxiomC{#1} \AxiomC{#2} \BinaryInfC{#3} \DisplayProof}

    % no indent
    \setlength\parindent{0pt}

    % reasoning proofs
    \newcommand{\proofline}[3]{(#1)\ & #2 & \text{#3} \\}
    \allowdisplaybreaks

    % actual document
    \begin{document}
        \section*{CO142 - Discrete Structures}
        \subsection*{9th October 2018}
        \subsubsection*{Recommended Books}
        \begin{itemize}
            \item K.H. Rosen. \textit{Discrete Mathematics and its Applications}
            \item J.L. Gersting. \textit{Mathematical Structures for Computer Science}
            \item J.K. Truss. \textit{Discrete Mathematics for Computer Science}
            \item R. Johsonbaugh. \textit{Disctete Mathematics}
            \item C. Schumacher. \textit{Fundamental Notions of Abstract Mathematics}
        \end{itemize}
        However, these books don't cover the same content. Learn his notation.
        \subsubsection*{Logical Formula, and Notation}
        This notation will be shared with \textbf{CO140}.
        \medskip

        $A \land B$ \hfill $A$ and $B$ both hold
        \smallskip

        $A \lor B$ \hfill $A$ or $B$ holds (or both)
        \smallskip

        $\neg A$ \hfill $A$ does not hold
        \smallskip

        $A \Rightarrow B$ \hfill if $A$ holds, then so does $B$
        \smallskip

        $A \Leftrightarrow B$ \hfill $A$ holds if and only if $B$ holds
        \smallskip

        $\forall x (A)$ \hfill the predicate $A$ holds for all $x$
        \smallskip

        $\exists x (A)$ \hfill the predicate $A$ holds for some $x$
        \smallskip

        $a \in A$ \hfill the object $a$ is in the set $A$ ($a$ is an element of $A$)
        \smallskip

        $a \notin A$ \hfill the object $a$ is not in the set $A$
        \smallskip

        $=_A$ \hfill tests whether two elements of $A$ are the same
        \subsubsection*{Sets}
        Sets are like data types in Haskell: Haskell data type declaration;
        \medskip

        \texttt{data Bool = False | True}
        \smallskip

        Set of Boolean values: \hfill \texttt{\{false, true\}}
        \smallskip

        List of Boolean values: \hfill \texttt{[true, false, true, false]}
        \smallskip

        Set equality: \hfill $\texttt{\{false, true\} = \{true, false\}}_\text{note order doesn't matter}$
        \medskip

        A set is a collection of objects from a pool of objects. Each object is an \textit{element}, or a \textit{member} of the set. A set \textit{contains} its elements. Sets can be defined in the following ways;
        \medskip

        $\{a_1, ..., a_2\}$ \hfill as a collection of $n$ distinct elements
        \smallskip

        $\{x \in A\ |\ P(x)\}$ \hfill for all the elements in $A$, where $P$ holds
        \smallskip

        $\{x\ |\ P(x)\}$ \hfill for all elements, where $P$ holds (dangerous - Russel's paradox)
        \subsubsection*{Use of "triangleq"}
        The use of $\triangleq$ is for "is defined by". Hence the empty set, $\varnothing \triangleq \{\}$. The difference between $\triangleq$ and $=$, is that the former cannot be proven, it is fact, whereas the latter takes work to prove.

        \subsubsection*{Russel's paradox}
        Not everything we write as $\{x\ |\ P(x)\}$ is automatically a set. Assume $R = \{X\ |\ X \notin X\}$ is a set, the set of all sets which don't contain themselves. As $R$ is a set, then $R \in R$, or $R \notin R$ (law of excluded middle), and thus we can do a case by case analysis.

        \begin{itemize}
            \item Assume $R \in R$. By the definition of $R$, it then follows that $R \notin R$ (if $R \in R$, then it doesn't satisfy the definition of $R$) - which is a contradiction.
            \item Assume $R \notin R$. It then follows that $R \in R$, as it follows the definition of $R$, hence it is another contradiction.
        \end{itemize}

        As both assumptions lead to contradictions, it's possible to write sets which aren't defined. We should only select from a set that we know is defined; $\{x \in A\ |\ P(x)\}$ - where $A$ is a set.
    \end{document}
