\documentclass[a4paper, 12pt]{article}
% packages
\usepackage{amssymb}
\usepackage[fleqn]{mathtools}
\usepackage{tikz}
\usepackage{enumerate}
\usepackage{bussproofs}
\usepackage{xcolor}
\usepackage[margin=1.3cm]{geometry}
\usepackage{logicproof}
\usepackage{diagbox}
\usepackage{listings}
\usepackage{graphicx}
\usepackage{lstautogobble}
\usepackage{hyperref}
\usepackage{multirow}
\usepackage{stmaryrd}
\usetikzlibrary{arrows, shapes.gates.logic.US, circuits.logic.US, calc, automata, positioning}

% shorthand for verbatim
\catcode`~=\active
\def~#1~{\texttt{#1}}

% code listing
\lstdefinestyle{main}{
    numberstyle=\tiny,
    breaklines=true,
    showspaces=false,
    showstringspaces=false,
    tabsize=2,
    numbers=left,
    basicstyle=\ttfamily,
    columns=fixed,
    fontadjust=true,
    basewidth=0.5em,
    autogobble,
    xleftmargin=3.0ex,
    mathescape=true
}
\newcommand{\dollar}{\mbox{\textdollar}} %
\lstset{style=main}

% augmented matrix
\makeatletter
\renewcommand*\env@matrix[1][*\c@MaxMatrixCols c]{%
\hskip -\arraycolsep
\let\@ifnextchar\new@ifnextchar
\array{#1}}
\makeatother

% ceiling / floor
\DeclarePairedDelimiter{\ceil}{\lceil}{\rceil}
\DeclarePairedDelimiter{\floor}{\lfloor}{\rfloor}

% custom commands
\newcommand{\indefint}[2]{\int #1 \, \mathrm{d}#2}
\newcommand{\defint}[4]{\int_{#1}^{#2} #3 \, \mathrm{d}#4}
\newcommand{\dif}[2]{\frac{\mathrm{d}#1}{\mathrm{d}#2}}
\newcommand{\limit}[2]{\displaystyle{\lim_{#1 \to #2}}}
\newcommand{\summation}[3]{\sum\limits_{#1}^{#2} #3}
\newcommand{\intbracket}[3]{\left[#3\right]_{#1}^{#2}}
\newcommand{\ulsmash}[1]{\underline{\smash{#1}}}

\newcommand{\powerset}[0]{\wp}
\renewcommand{\emptyset}[0]{\varnothing}

\newcommand{\unaryproof}[2]{\AxiomC{#1} \UnaryInfC{#2} \DisplayProof}
\newcommand{\binaryproof}[3]{\AxiomC{#1} \AxiomC{#2} \BinaryInfC{#3} \DisplayProof}
\newcommand{\trinaryproof}[4]{\AxiomC{#1} \AxiomC{#2} \AxiomC{#3} \TrinaryInfC{#4} \DisplayProof}

% no indent
\setlength\parindent{0pt}
\setlength\itemsep{0em}

% reasoning proofs
\usepackage{ltablex}
\usepackage{environ}
\keepXColumns
\NewEnviron{reasoning}{
    \begin{tabularx}{\textwidth}{rlX}
        \BODY
    \end{tabularx}
}
\newcommand{\proofline}[3]{$(#1)$ & $#2$ & \hfill #3 \smallskip \\}
\newcommand{\proofarbitrary}[1]{& take arbitrary $#1$ \smallskip \\}
\newcommand{\prooftext}[1]{\multicolumn{3}{l}{#1} \smallskip \\}
\newcommand{\proofmath}[3]{$#1$ & = $#2$ & \hfill #3 \smallskip \\}
\newcommand{\prooftherefore}[1]{& $\therefore #1$ \smallskip \\}
\newcommand{\proofbc}[0]{\prooftext{\textbf{Base Case}}}
\newcommand{\proofis}[0]{\prooftext{\textbf{Inductive Step}}}

% reasoning er diagrams
\newcommand{\nattribute}[4]{
    \node[draw, state, inner sep=0cm, minimum size=0.2cm, label=#3:{#4}] (#1) at (#2) {};
}
\newcommand{\mattribute}[4]{
    \node[draw, state, accepting, inner sep=0cm, minimum size=0.2cm, label=#3:{#4}] (#1) at (#2) {};
}
\newcommand{\dattribute}[4]{
    \node[draw, state, dashed, inner sep=0cm, minimum size=0.2cm, label=#3:{#4}] (#1) at (#2) {};
}
\newcommand{\entity}[3]{
    \node[] (#1-c) at (#2) {#3};
    \node[inner sep=0cm] (#1-l) at ($(#1-c) + (-1, 0)$) {};
    \node[inner sep=0cm] (#1-r) at ($(#1-c) + (1, 0)$) {};
    \node[inner sep=0cm] (#1-u) at ($(#1-c) + (0, 0.5)$) {};
    \node[inner sep=0cm] (#1-d) at ($(#1-c) + (0, -0.5)$) {};
    \draw
    ($(#1-c) + (-1, 0.5)$) -- ($(#1-c) + (1, 0.5)$) -- ($(#1-c) + (1, -0.5)$) -- ($(#1-c) + (-1, -0.5)$) -- cycle;
}
\newcommand{\relationship}[3]{
    \node[] (#1-c) at (#2) {#3};
    \node[inner sep=0cm] (#1-l) at ($(#1-c) + (-1, 0)$) {};
    \node[inner sep=0cm] (#1-r) at ($(#1-c) + (1, 0)$) {};
    \node[inner sep=0cm] (#1-u) at ($(#1-c) + (0, 1)$) {};
    \node[inner sep=0cm] (#1-d) at ($(#1-c) + (0, -1)$) {};
    \draw
    ($(#1-c) + (-1, 0)$) -- ($(#1-c) + (0, 1)$) -- ($(#1-c) + (1, 0)$) -- ($(#1-c) + (0, -1)$) -- cycle;
}

% actual document
\begin{document}
    \section*{CO142 - Discrete Structures}
        \subsection*{Prelude}
            The content discussed here is part of CO142 - Discrete Structures (Computing MEng); taught by Steffen van Bakel, in Imperial College London during the academic year 2018/19. The notes are written for my personal use, and have no guarantee of being correct (although I hope it is, for my own sake). This should be used in conjunction with the (extremely detailed) notes.
        \subsection*{9th October 2018}
            \subsubsection*{Recommended Books}
                \begin{itemize}
                    \itemsep0em
                    \item K.H. Rosen. \textit{Discrete Mathematics and its Applications}
                    \item J.L. Gersting. \textit{Mathematical Structures for Computer Science}
                    \item J.K. Truss. \textit{Discrete Mathematics for Computer Science}
                    \item R. Johsonbaugh. \textit{Discrete Mathematics}
                    \item C. Schumacher. \textit{Fundamental Notions of Abstract Mathematics}
                \end{itemize}
                However, these books don't cover the same content. Learn his notation.
            \subsubsection*{Logical Formula, and Notation}
                This notation will be shared with \textbf{CO140}.
                \begin{itemize}
                    \itemsep0em
                    \item $A \land B$ \hfill $A$ and $B$ both hold
                    \item $A \lor B$ \hfill $A$ or $B$ holds (or both)
                    \item $\neg A$ \hfill $A$ does not hold
                    \item $A \Rightarrow B$ \hfill if $A$ holds, then so does $B$
                    \item $A \Leftrightarrow B$ \hfill $A$ holds if and only if $B$ holds
                    \item $\forall x (A)$ \hfill the predicate $A$ holds for all $x$
                    \item $\exists x (A)$ \hfill the predicate $A$ holds for some $x$
                    \item $a \in A$ \hfill the object $a$ is in the set $A$ ($a$ is an element of     \item $A$)
                    \item $a \notin A$ \hfill the object $a$ is not in the set $A$
                    \item $=_A$ \hfill tests whether two elements of $A$ are the same
                \end{itemize}
            \subsubsection*{Sets}
                Sets are like data types in Haskell: Haskell data type declaration;
                \begin{itemize}
                    \itemsep0em
                    \item \texttt{data Bool = False | True}
                    \item \texttt{\{false, true\}} \hfill set of boolean values
                    \item \texttt{[true, false, true, false]} \hfill list of boolean values
                    \item \texttt{\{false, true\} = \{true, false\}} \hfill set equality (note that order doesn't matter)
                \end{itemize}
                A set is a collection of objects from a pool of objects. Each object is an \textit{element}, or a \textit{member} of the set. A set \textit{contains} its elements. Sets can be defined in the following ways;
                \begin{itemize}
                    \itemsep0em
                    \item $\{a_1, ..., a_2\}$ \hfill as a collection of $n$ distinct elements
                    \item $\{x \in A \mid P(x)\}$ \hfill for all the elements in $A$, where $P$ holds
                    \item $\{x \mid P(x)\}$ \hfill for all elements, where $P$ holds (dangerous - Russel's paradox)
                \end{itemize}
            \subsubsection*{Use of "triangleq"}
                The use of $\triangleq$ is for "is defined by". Hence the empty set, $\varnothing \triangleq \{\}$. The difference between $\triangleq$ and $=$, is that the former cannot be proven, it is fact, whereas the latter takes work to prove.
            \subsubsection*{Russel's paradox}
                Not everything we write as $\{x \mid P(x)\}$ is automatically a set. Assume $R = \{X \mid X \notin X\}$ is a set, the set of all sets which don't contain themselves. As $R$ is a set, then $R \in R$, or $R \notin R$ (law of excluded middle), and thus we can do a case by case analysis.
                \begin{itemize}
                    \itemsep0em
                    \item Assume $R \in R$. By the definition of $R$, it then follows that $R \notin R$ (if $R \in R$, then it doesn't satisfy the definition of $R$) - which is a contradiction.
                    \item Assume $R \notin R$. It then follows that $R \in R$, as it follows the definition of $R$, hence it is another contradiction.
                \end{itemize}
                As both assumptions lead to contradictions, it's possible to write sets which aren't defined. We should only select from a set that we know is defined; $\{x \in A \mid P(x)\}$ - where $A$ is a well-defined set.
        \subsection*{12th October 2018}
            \subsubsection*{Set Comparisons}
                We can define a set $A$, as being a subset of another set $B$ if every element in $A$ is an element in $B$. This can be formally written as; $A \subseteq B \triangleq \forall x \in A (x \in B)$. Note that we can also say $\forall x (x \in A \Rightarrow x \in B)$, and the two hold the same meaning. It's important to clarify in the latter that we're not the domain of $x$, an we assume there is a universe of possible objects which forms a set. We're also able to define a strict subset such that $A \subset B \triangleq A \subseteq B \land A \neq B$.
                \medskip

                We can say that any set is a trivial subset of itself, as we'd have $x \in A \Rightarrow x \in A$, which always evaluates to true, from propositional logic. Another trivial example is that $\emptyset$, the empty set, is a subset of every set. Using the second definition of subset, we can say that as $x \in \emptyset$ is false, by definition, and anything follows from falsity, whereas in the first definition we argue that all (0) elements of $\emptyset$ are in some other set.
                \medskip

                We can also define set equality as $A = B \triangleq A \subseteq B \land B \subseteq A$. However, we can also consider the set composition notation for a set, such that $A = \{x \mid P(x)\}$, and $B = \{x \mid Q(x)\}$. If we're able to prove that $\forall x (P(x) \Leftrightarrow Q(x))$, it follos that $A = B$. This method can be quite powerful if we're familiar with logic, and equivalences. We can justify this by saying that $y \in A \Rightarrow P(y) \Rightarrow Q(y) \Rightarrow y \in B$, and also in the other direcion; $y \in B \Rightarrow Q(y) \Rightarrow P(y) \Rightarrow y \in A$.
            \subsubsection*{Set Composition}
                \begin{itemize}
                    \itemsep0em
                    \item $A \cup B \triangleq \{x \mid x \in A \lor x \in B\}$ \hfill set union
                    \item $A \cap B \triangleq \{x \in A \mid x \in B\}$ \hfill set intersection
                    \item $A \backslash B$ (or $A - B$) $\triangleq \{x \in A \mid x \notin B\}$ \hfill set difference
                    \item $A \triangle B \triangleq (A \backslash B) \cup (B \backslash A)$ \hfill symmetric set difference))
                    \item $A \cap B = \emptyset$ \hfill disjoint set
                \end{itemize}
            \subsubsection*{A Note on Proofs}
                Instead of writing out the formal definition, where we may lose the intuition, using a natural language (direct) proof is acceptable in this course.
                \medskip

                Consider the following proof; $A \subseteq B$, and $B \subseteq C$, then show $A \subseteq C$. Here, we want to show that any element of $A$, is also an element of $C$. We can approach this intuitively by taking an arbitrary $a \in A$. By the the first assumption, we can say $a \in B$. Then, by the second assumption, $a \in C$. However, we've taken an arbitrary $a$, therefore this follows $\forall a \in A (a \in C)$, therefore $A \subseteq C$.
                \medskip

                The crucial part of the aforementioned proof is the use of some \textbf{arbitrary} value. If we were to do a proof on the natural numbers, to show $\forall n \in \mathbb{N} [\text{even}(n)]$, and we proved even(2), it wouldn't prove it for all natural numbers.
                \medskip

                We also want to aim for a direct proof, instead of a proof by contradiction, since we will often do the following; assume $\neg A$, then we somehow get $A$, which causes a contradiction ($\lightning$), and therefore $A$. However, we still did all the work to prove $A$.
                \medskip

                Consider the proof to show that $C \cap D = D \cap C$. Let us first take some arbitary $x \in (C \cap D)$. By definition of union, we know that $x \in C$, and $x \in D$. Therefore, it also fits the predicate for $(D \cap C)$. As such, $C \cap D \subseteq D \cap C$. To prove the other direction is trivial, and almost identical to this direction. Since we've proved both directions of $\subseteq$, we can conclude equality.


\end{document}