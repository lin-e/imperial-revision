\documentclass[a4paper, 12pt]{article}
% packages
\usepackage{amssymb}
\usepackage[fleqn]{mathtools}
\usepackage{tikz}
\usepackage{enumerate}
\usepackage{bussproofs}
\usepackage{xcolor}
\usepackage[margin=1.3cm]{geometry}
\usepackage{logicproof}
\usepackage{diagbox}

% augmented matrix
\makeatletter
\renewcommand*\env@matrix[1][*\c@MaxMatrixCols c]{%
\hskip -\arraycolsep
\let\@ifnextchar\new@ifnextchar
\array{#1}}
\makeatother

% ceiling / floor
\DeclarePairedDelimiter{\ceil}{\lceil}{\rceil}
\DeclarePairedDelimiter{\floor}{\lfloor}{\rfloor}

% custom commands
\newcommand{\indefint}[2]{\int #1 \, \mathrm{d}#2}
\newcommand{\defint}[4]{\int_#1^#2 #3 \, \mathrm{d}#4}
\newcommand{\dif}[2]{\frac{\mathrm{d}#1}{\mathrm{d}#2}}
\newcommand{\limit}[2]{\displaystyle{\lim_{#1 \to #2}}}
\newcommand{\summation}[3]{\sum\limits_{#1}^#2 #3}
\newcommand{\intbracket}[3]{\left[#3\right]_#1^#2}

\newcommand{\powerset}[0]{\wp}
\renewcommand{\emptyset}[0]{\varnothing}

\newcommand{\unaryproof}[2]{\AxiomC{#1} \UnaryInfC{#2} \DisplayProof}
\newcommand{\binaryproof}[3]{\AxiomC{#1} \AxiomC{#2} \BinaryInfC{#3} \DisplayProof}

% no indent
\setlength\parindent{0pt}

% reasoning proofs
\newcommand{\proofline}[3]{(#1)\ & #2 & \text{#3} \\}
\allowdisplaybreaks

% actual document
\begin{document}
    \section*{CO112 - Hardware}
        \subsection*{Prelude}
            The content discussed here is part of CO112 - Hardware (Computing MEng); taught by Bernhard Kainz, and Bjoern Schuller, in Imperial College London during the academic year 2018/19. The notes are written for my personal use, and have no guarantee of being correct (although I hope it is, for my own sake). This should be used in conjunction with the notes, and lecture slides.
        \subsection*{Notes - Lecture 1}
            This section will be covered in less detail, as we've gone through. However, we will need to change the notation we use in this course from the one used in logic, from using $\land$ to $\cdot$, $\lor$ to $+$, and from $\neg$ to $^\prime$.
            \begin{center}
                \begin{tabular}{cc|c|c|c}
                    $A$ & $B$ & $A \cdot B$ (AND) & $A + B$ (OR) & $A^\prime$ (NOT) \\
                    \hline
                    0 & 0 & 0 & 0 & 1 \\
                    0 & 1 & 0 & 1 & 1 \\
                    1 & 0 & 0 & 1 & 0 \\
                    1 & 1 & 1 & 1 & 0
                \end{tabular}
            \end{center}
        \subsection*{Notes - Lecture 14}
        

\end{document}
