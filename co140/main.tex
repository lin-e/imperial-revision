\documentclass[a4paper, 12pt]{article}
    % packages
    \usepackage{amssymb}
    \usepackage[fleqn]{mathtools}
    \usepackage{tikz}
    \usepackage{enumerate}
    \usepackage{bussproofs}
    \usepackage[margin=1.3cm]{geometry}

    % augmented matrix
    \makeatletter
    \renewcommand*\env@matrix[1][*\c@MaxMatrixCols c]{%
    \hskip -\arraycolsep
    \let\@ifnextchar\new@ifnextchar
    \array{#1}}
    \makeatother

    % ceiling / floor
    \DeclarePairedDelimiter{\ceil}{\lceil}{\rceil}
    \DeclarePairedDelimiter{\floor}{\lfloor}{\rfloor}

    % custom commands
    \newcommand{\indefint}[2]{\int #1 \, \mathrm{d}#2}
    \newcommand{\defint}[4]{\int_#1^#2 #3 \, \mathrm{d}#4}
    \newcommand{\dif}[2]{\frac{\mathrm{d}#1}{\mathrm{d}#2}}
    \newcommand{\limit}[2]{\displaystyle{\lim_{#1 \to #2}}}
    \newcommand{\summation}[3]{\sum\limits_{#1}^#2 #3}
    \newcommand{\intbracket}[3]{\left[#3\right]_#1^#2}

    \newcommand{\powerset}[0]{\wp}
    \renewcommand{\emptyset}[0]{\varnothing}

    \newcommand{\unaryproof}[2]{\AxiomC{#1} \UnaryInfC{#2} \DisplayProof}
    \newcommand{\binaryproof}[3]{\AxiomC{#1} \AxiomC{#2} \BinaryInfC{#3} \DisplayProof}

    % no indent
    \setlength\parindent{0pt}

    % reasoning proofs
    \newcommand{\proofline}[3]{(#1)\ & #2 & \text{#3} \\}
    \allowdisplaybreaks

    % actual document
    \begin{document}
        \section*{CO140 - Logic}
        \subsection*{Introduction}
        A logic system consists of 3 things:
        \begin{enumerate}[1.]
            \item Syntax - formal language used to express concepts
            \item Semantics - meaning for the syntax
            \item Proof theory - syntactic way of identifying valid statements of language
        \end{enumerate}
        Considering the basic example in a program, we can then see the features;
        \begin{verbatim}
if count > 0 and not found then
    decrement count;
    look for next entry;
end if
        \end{verbatim}
        \begin{enumerate}[1.]
            \item basic (\textbf{atomic}) statements (\textbf{propositions}) are either $\top$ or $\bot$ depending on circumstance;
                \begin{enumerate}[i.]
                    \item \texttt{count > 0}
                    \item \texttt{found}
                \end{enumerate}
            \item \textbf{boolean operations}, such as \texttt{and}, \texttt{or}, \texttt{not}, etc. are used to build complex statements from \textbf{atomic propositions}
            \item the final statement \texttt{count > 0 and not found} evalulates to either $\top$ or $\bot$
        \end{enumerate}
        \subsection*{Syntax}
        The formal language of logic consists of three ingredients;
        \begin{enumerate}[1.]
            \item Propositional atoms (propositional variables), evaluate to a truth value of either $\top$ or $\bot$. These are represented with letters; $p, p^\prime, p_0, p_1, p_2, p_n, q, r, s, ...$
            \item Boolean connectives;
                \begin{itemize}
                    \item 
                \end{itemize}
        \end{enumerate}
    \end{document}
